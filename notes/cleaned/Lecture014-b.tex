\section{Solutions of Nonlinear Systems}

The Basic Problem:

Find a root of $x\in\mathbb{R}$ of an equation of the form $f(x)=0$ for a
given continuous function $f$.

\section{The Bisection Method}

In most cases it is not really possible to solve analytically. We will consider
iterative methods to approximate the solution. Our first method will be the
\uline{Bisection Method}. We must start with an interval $[a, b]$ with 
$f(a)$ and $f(b)$ of opposite signs.

By the intermediate value theorem, there exists a number $c$ in $(a, b)$ such that
$f(c)=0$.

\thm If $f\in C[a,b]$ and $k$ is any number between $f(a)$ and $f(b)$, then
there exists a number $c$ in $(a, b)$ such that $f(c)=k$.

Now set $a_1 = a$ and $b_1 = b$ and let $p_1$ be the midpoint of $[a_1, b_1]$.

\begin{enumerate}
    \item Compute the midpoint:
    \begin{equation*}
        p_1 = \frac{1}{2} (a_1 + b_1)
    \end{equation*}
    
    \item If $f(p_1) = 0$, then we are done. Set $p = p_1$.
    
    \item If $f(p_1)$ and $f(a_1)$ are of opposite signs, then there must exist 
          a root $p \in (a_1, p_1)$ such that $f(p) = 0$. Set $a_2 = a_1$ and 
          $b_2 = p_1$.
    
    \item Otherwise, if $f(p_1)$ and $f(b_1)$ are of opposite signs, then there 
          must exist a root $p \in (p_1, b_1)$ such that $f(p) = 0$. Set 
          $a_2 = p_1$ and $b_2 = b_1$.
    
    \item Reapply the process to the new interval $[a_2, b_2]$.
    
    \item Once the stopping criteria are satisfied, set the midpoint of the 
          interval as the estimate for the root.
\end{enumerate}

\subsection{Possible Stopping Criteria}

\begin{enumerate}
  \item $\displaystyle \frac{b_n-a_n}{2} < \text{ TOL}$ \quad\uline{or} \quad
    $\displaystyle \left| p_n - p_{n-1} \right| < \text{ TOL}$
    \subitem GOOD: Ensures that the returned root value $p_n$ is within 
    tolerance of the exact value $p$
    \subitem GOOD: Easy error analysis
    \subitem BAD: Does not ensure that $f(p_n)$ is small.
    \subitem BAD: An absolute rather than relative measure of error.
  \item $\displaystyle \frac{|p_n - p_{n-1}}{p_n} < \text{ TOL}, p_n \neq 0$
    \subitem Usually preferred over $(1)$ if nothing is known about $f(\cdot)$ or
    $p$
  \item $\left| f(p_n) \right| < \text{ TOL}$
    \subitem Ensures that $f(p_n)$ is small, but $p_n$ may differ significantly
    from the true root $p$.
  \item We can also carry out a fixed number of iterations $N$-- This is closely
    related to $(1)$
\end{enumerate}

The best stopping criteria will depend on what is known about $f$ and $p$ and
on the type of problem. It's often useful to use criteria 2 (relative error test)
and criteria 4 (fixed number of steps) together.

