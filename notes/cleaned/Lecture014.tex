\lemma If the spectral radius $\rho(T)$ satisfies $\rho(T) < 1$ then 
$(I-T)^{-1} = I + T + T^2 + \dots$

And we will prove the following theorem:

\thm For any $x^{(0)} \in \mathbb{R}^n, \left\{ x^{(k)} \right\}_{k=0}^\infty$
the sequence defined by 

\begin{equation*}
  x^{(k)} = Tx^{(k-1)} + c
\end{equation*}

converges to the unique solution of 

\begin{equation*}
  x = Tx + c \text{ if and only if }  \rho(T)<1
\end{equation*}

\bproof: assume $\rho(T)<1$

\begin{align*}
  x^{(k)} &= Tx^{(k-1)} + c \\
  &= T(Tx^{(k-2)} + c) + c \\
  &= T^2x^{(k-2)} + (T+I)c \\
  \vdots \\
  &= T^kx^{(0)} + (T^{k-1}+\dots+T+I)c 
\end{align*}

Since $\rho(T)<1$, the matrix $T$ is convergent and

\begin{equation*}
  \lim_{k \to \infty} T^kx^{(0)} = 0
\end{equation*}

\fproof

HAS NOT BEEN WRITTEN DOWN YET

The \lemma implies that 

\begin{equation*}
  \lim_{k \to \infty} x^{(k)} = \lim_{k \to \infty} T^kx^{(0)} + \lim_{k \to \infty} \left(\sum_{j=0}^{k-1} T^j \right)c = 0 + (1-T)^{-1}c
\end{equation*}

$\implies \left\{ x^{(k)} \right\}$ converges to the unique solution of $x=Tx+c$

\ie $(I-T)x = c \implies x = (I-T)^{-1}c$

This allows us to derive some related results on the rates of convergence.

\textbf{Corollary}: If $||T||<1$ for any natural matrix norm and $c$ is a given
vector, then the sequence $\left\{ x^{(k)} \right\}_{k=0}^\infty$ defined by

\begin{equation*}
  x^{(k)} = Tx^{(k-1)} + c
\end{equation*}

converges for any $x^{(0)} \in \mathbb{R}^n$ to a vector $x \in \mathbb{R}^n$
and the following error bounds hold:

\begin{enumerate}[label=(\roman*)]
  \item $||x-x^{(k)}||\leq ||T|^k ||x^{(0)}-x||$
  \item $||x-x^{(k)}|| \leq \frac{||T||^k}{-||T||}||x^{(1)}-x^{(0)}||$
\end{enumerate}

Note, however, that $\rho(A) \leq ||A||$ for \uline{any} natural norm. In practice,

\begin{equation*}
  ||x-x^{(k)}|| \approx \rho(T)^k ||x^{(0)}-x||
\end{equation*}

so it is desirable to have $\rho(T)$ as small as possible.

Some results for Jacobi's and Gauss-Seidel methods:

\thm If $A$ is strictly diagonally dominant, then for any choice of $x^{(0)}$,
both the Jacobi and Gauss-Seidel methods give sequences $\left\{ x^{(k)} \right\}_{k=0}^\infty$
that converge to the unique solution of $Ax=b$.

No general results exist tot tell which of the two methods will converge more 
quickly, but the following result applies in a variety of examples:

\thm Stein Rosenberge

If $a_{ij} \leq 0$ for each $i\ne j$ and $a_{ii} >0$ for each $i=1,2,\dots,n$, 
then exactly one of the following holds.

\begin{enumerate}[label=(\alph*)]
  \item $0 \leq \rho(T_g) < \rho(T_j) < 1$
\end{enumerate}

\section{Successive Over-Relaxation (SOR)}

To define, suppose $\tilde{x}^{(k+1)}$ is the iterate from Gauss-Seidel using 
$x^{(k)}$ as the initial guess. The $(k+1)^{st}$ iterate of SOR is defined by

\begin{equation*}
  x^{(k+1)} = \omega \tilde{x} ^{(k)} + (1-\omega) x^{(k)}
\end{equation*}

where $1<\omega<2$. It can be difficult to select $\omega$ optimally. Indeed, the answer
to this question is not known for general $n \times n$ linear systems.

However, we do have the following results:

\thm (kahan): If $a_{ii} \ne 0$ for each $i$, then

\begin{equation*}
  \rho(T_{SOR}) \geq |\omega-1|
\end{equation*}

$\implies$ SOR can converge only if $0< \omega < 2$

\thm (ostrowski-reich): If $A$ is a positive definite matrix and $0<\omega<2$, then 
the SOR method converges for any choice of initial approximate vector $x^{(0)}$

\thm. If $A$ is positive definite and tridiagonal, then 

\begin{equation*}
  \rho(T_g) = \bqty{
    \rho(T_j)
  }^2 < 1
\end{equation*}

and the optimal choice of $\omega$ for the SOR method is 

\begin{equation*}
  \omega = \frac{2}{1+\sqrt{1-\rho(T_j)^2}}
\end{equation*}

with this choice of $\omega$, we have $\rho(T_{SOR}) = \omega - 1$

