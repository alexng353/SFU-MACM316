\newcommand{\fl}{\operatorname{fl}}

\section{Floating Point Decimal Normalization}

Can we write all real numbers in normalized scientific notation? Well, the short
answer is yes.
\begin{align*}
    732.5051 &\rightarrow +0.7325051 \times 10^{+3} \\
    -0.005612 &\rightarrow -0.5612 \times 10^{-2}
\end{align*}
For $x \in \mathbb{R}$, we can express it as:
\begin{equation*}
    x = \pm r \times 10^{\pm n}, \quad \text{where } \frac{1}{10} \leq r \leq 1.
\end{equation*}
In binary, we write:
\begin{equation*}
  e x = \pm q \times 2^{\pm m}, \quad \text{where } \frac{1}{2} \leq q < 1.
\end{equation*}
Here, $q$ is the mantissa and $m$ is the integer exponent.

Let $b$ be the base.
We limit $r$ and $q$ so that when $k < 1/b$, we can shift the decimal 
place and normalize the number further. When we have $1.x$, we rewrite it 
as $0.x \times b^1$.

\section{What if we have too many digits?}
\subsection{Rounding}
% \section{Rounding and Chopping}
 % (Sources of Error)
Given $x = 0.a_1 a_2 \dots a_n a_{n+1} \dots a_m$ using $m$ digits, rounding 
to $n$ places follows:
\begin{itemize}
    \item If $0 \leq a_{n+1} < 5$, then $x = 0.a_1 a_2 \dots a_n$.
    \item If $5 \leq a_{n+1} \leq 9$, then $x = 0.a_1 a_2 \dots (a_n + 1)$.
\end{itemize}
As you will see in the following example, this definition of rounding is
different from the way we traditionally round numbers. We only consider the
magnitude of the smallest significant digit, not the sign, so that when you
round a number and its additive inverse, they will still cancel out.

\subsubsection{Example}
\begin{align*}
    \text{round}(0.125) &= 0.13, \\
    \text{round}(-0.125) &= -0.13.
\end{align*}

\subsection{Chopping (Truncation)}
Compared to rounding, truncation/chopping to $n$ decimal places follows:
\begin{align*}
  x &= 0.a_1 a_2 \dots a_na_{n+1} \dots a_m, \\
  \text{chop}_n(x) &= 0.a_1 a_2 \dots a_n.
\end{align*}

Truncation introduces larger errors but is computationally cheaper than 
rounding. In general, we prefer rounding because of the higher accuracy, because
the computational cost is generally considered negligible.

\section{Error}

\subsection{Definition}

We define:
\begin{itemize}
  \item Actual error: $p - \hat{p}$.
    \item Absolute error: $\abs{p - \hat{p}}$.
    \item Relative error: $\frac{\abs{p - \hat{p}}}{\abs{p}}$.
\end{itemize}
\tiny Note that the notes use $p$ and $p^*$ but I will use them interchangeable with
$p$ and $\hat{p}$.

\normalsize
Absolute error is used when magnitude matters, particularly for small values. 
Relative error is preferred when values differ in scale.

\subsection{Significant Digits}
An approximation $\hat p$ has $t$ significant digits if

\begin{equation*}
  \frac{\abs{p-\hat p}}{\abs{p}} \leq 5 \times 10^{-t}
\end{equation*}

\subsection{Example}
\begin{align*}
    \text{Exact: } & 0.1, \quad \text{Approximate: } 0.099, \\
    \text{Relative Error: } & \frac{|0.1 - 0.099|}{0.1} = 0.01.
\end{align*}

\begin{center}
\begin{tabular}{|c|c|c|}
    \hline
    $t$ & $5 \times 10^{-t}$ & \text{Is error within bound?} \\
    \hline
    0 & 5 & $\checkmark$ \\
    1 & 0.5 & $\checkmark$ \\
    2 & 0.05 & $\checkmark$ \\
    3 & 0.005 & $\times$ \\
    \hline
\end{tabular}
\end{center}

Since $0.01 < 5 \times 10^{-2}$ but not $5 \times 10^{-3}$, we have two 
significant digits.

\section{Computations and Machine Representation}

Let $\fl(x)$ denote the machine representation of $x$. Computations on a 
machine follow:
\begin{equation*}
    \fl(\fl(x) + \fl(y)).
\end{equation*}
Each step introduces an error.

\subsection{Example}
\begin{align*}
    p &= 0.54617, \quad q = 0.54601, \\
    r &= p - q = 0.00016.
\end{align*}
With 4-digit rounding,
\begin{align*}
    p^* &= 0.5462, \quad q^* = 0.5460, \\
    r^* &= p^* - q^* = -0.0002.
\end{align*}
Relative error:
\begin{equation*}
    \frac{|r - r^*|}{|r|} = 0.25.
\end{equation*}
A high relative error results when subtracting close numbers.

\section{Roundoff Error} % continued into Lecture 003

Consider computing $f(x) = \frac{1 - \cos x}{x^2}$ for $\bar{x} = 1.2 \times 10^{-5}$.
With 10-digit rounding:
\begin{align*}
    c &= \fl(\cos \bar{x}) = 0.9999999999, \\
    1 - c &= 0.0000000001.
\end{align*}
This results in a large error.
Instead, if we use an alternative formula, such as $\cos x = 1 - 2\sin^2(x/2)$:
\begin{equation*}
    f(x) = \frac{1}{2} \left( \frac{\sin(x/2)}{x/2} \right)^2.
\end{equation*}
We obtain a more accurate computation.

\textbf{Conclusion:} Avoid subtracting close numbers. Use alternative 
representations like Taylor series, trigonometric identities or rationalized 
approximations.

