\section{Initial Value Problems for ordinary Differential Equations}

Many natural, scientific and engineering problems can be described in terms of 
\uline{differential equations}. Differential equations give us a way to
mathematically expres rates of change. We will be considering methods for
treating ordinary differential equations (ODEs) in this lecture. ODEs only
consider derivatives with respect to one variable.

\Ex Let $y(t)$ denote the number of individuals in a certain population. If this
population has a constant growth rate $\alpha$ (the difference between a
constant birth rate and death rate), then the differential equation 

\[
y'(t) = \alpha y(t)
.\]

with initial condition $y(0) = y_0$ describes the population growth.

\soln 

% \begin{align*}
%   \frac{y'}{y} &= \alpha \\
%   \frac{dy}{y} &= \alpha \, dx \\
%   \int \frac{dy}{y} &= \int \alpha \, dx \\
%   \ln |y| &= \alpha x + C \\
%   |y| &= e^{\alpha x + C} = e^C \cdot e^{\alpha x} \\
%   y &= C_1 e^{\alpha x} \quad \text{where } C_1 = \pm e^C
% \end{align*}

\begin{align*}
  \frac{y'(t)}{y(t)} &= \alpha \\
  D_t[\ln(y(t))] &= \alpha \\
  \ln(y(t)) &= \alpha t + \text{const}  \\
  y(t) &= ce^{\alpha t} \qquad c = e^{\text{const}}\\
\end{align*}

\begin{equation*}
  y(0) = y_0 \implies y(t) = y_0 e^{\alpha t}
.\end{equation*}

To include other effects such as overcrowding, competition for food, etc, one
might introduce a second term into the equation

\begin{equation*}
  y'(t) = \alpha y(t) - \beta [y(t)]^2
.\end{equation*}

where $\beta > 0$ and $\beta$ is small. Introducing a nonlinear term makes the
problem much more difficult to study analytically.

Indeed, few problems originating from the study of physical phenomena can be
solved exactly. We begin by studying numerical methods for approximating the
solution $y(t)$ to a problem.

\[
  \frac{dy}{dt} = f(t, y) \qquad \text{for } a \leq t \leq b
.\]

subject to the initial condition 

\[
y(a) = \alpha
.\]

\subsection{The elementary theory of initial value problems}

We want/need some theoretical results, in particular, we would like to show that
solutions to equations exist and are unique.

\defn A function $f(t,y)$ satisfies a Lipschitz condition in the variable $y$ on
a set $D \in \mathbb{R}^2$ if a constant $L > 0$ exists such that

\[
  |f(t,y_1) - f(t, y_2)| \leq L|y_1 - y_2| \qquad \text{for all } (t,y_1), (t,y_2) \in D
.\]

The constant $L$ is called a \uline{Lipschitz constant} for $f$.

\pagebreak
\subsubsection{Example}

Does $f(t,y) = ty$ satisfy a Lipschitz condition on 

\[
D  = \{(t,y) : 0 \leq t \leq 1, -\infty < y < \infty\}
?\]

\noindent
\soln

% \begin{align*}
% &|f(t,y_1) - f(t, y_2)| \\
% &= |-ty_1 + \frac{4t}{y_1} + ty_2 - \frac{4t}{y_2}|\\
% \end{align*}

\begin{equation*}
|f(t,y_1) - f(t, y_2)| = |-ty_1 + \frac{4t}{y_1} + ty_2 - \frac{4t}{y_2}|
.\end{equation*}


Consider $y_1 = -1, t=1$, $y_2 \to 0^+$. Under these conditions, we have
$|f(t,y_1) - f(t,y_2)| \to \infty$.

RHS = $L|y_1-y_2| \to L$

$L$ is finite.

We cannot have $|f(t,y_1) - f(t,y_2)| \leq L|y_1 - y_2|$ for any finite $L$

$\therefore$ Lipschitz condition does not hold.

