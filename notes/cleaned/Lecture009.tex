\renewcommand{\arraystretch}{1.25} % Adjust row spacing
\setlength{\arraycolsep}{12pt}

\section{LU Decomposition (contd.)}

THM: If Gaussian elimination can be performed without row exchanges, then the
matrix $A$ has a unique $LU$ factorization where $L$ is lower triangular and
with all diagonal entries equal to $1$ and $U$ is an upper triangular matrix.

Once the matrix factorization is complete, the solution to

\begin{equation*}
  LUx = b
\end{equation*}

is found by first setting 

\begin{equation*}
  y=Ux
\end{equation*}

then determining the vector $y$ from $Ly=b$ using forward substitution. The
variable $x$ is found from $Ux=y$ using backward substitution.

Notice: in the theorem, \textbf{we need to be able to perform Gaussian
Elimination without row exchanges}. This means that we can't use row
exchanges to solve the system.

\section{Matrix Factorization}

If $A$ is any nonsingular matrix, $Ax=b$ can be solved by Gaussian Elimination
with the possibility of row exchanges. If we can find the row interchanges
required to solve the system by Gaussian Elimination, then we can re-arrange
the equations in an order that would ensure that no row interchanges are
required.

$\implies$ There is a rearrangement of the equations that permits Gaussian
Elimination without row exchanges.

But WHY does the rearrangement of the equations break LU decomposition?

\section{Permutation Matrix}

An $n\times n$ permutation matrix $P$ is obtained by rearranging the rows of
the identity matrix. This gives a matrix with precisely one nonzero entry in
each row and column. The nonzero entries are all $1$'s.

On the other hand, multiplying $A$ on the right by $P$ will exchange the
second and third \textbf{columns} of $A$.

We will be using the following two properties of permutation matrices:

\begin{enumerate}
  \item If $K_1,...,K_n$ is a permutation of the integers $1,...,n$, and the
    permutation matrix $P=[p_{ij}]$ is defined by 

    \qquad $p_{ij}=\begin{cases}
      1 & i=K_j\\
      0 & \text{otherwise}
    \end{cases}$

    Then, $PA$ permutes the rows of $A$ according to (BIGASS MATRIX MISSING HERE)
    data from chapter6-part2.pdf page 3.
  \item If $P$ is a permutation matrix, then $P^{-1}$ exists and $P^{-1}=P^T$
\end{enumerate}

Now our approach will be to left multiply the system

\begin{equation*}
  Ax=b
\end{equation*}

by the appropriate permutiation matrix $P_1$ so that the system

\begin{equation*}
  (PA)x=Pb
\end{equation*}

can be solved without row exchanges. Then $PA$ can be factorized into

\begin{equation*}
  PA=LU
\end{equation*}

This tells us that

\begin{align*}
  P^{-1}LUx &= b \\
  LUx &= Pb
\end{align*}

which can be solved rapidly for $x$.



