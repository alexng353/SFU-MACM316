When we use iterative matrix techniques, we want to know when powers of a 
matrix become small.

\defn We call an $n \times n$ matrix $A$ convergent if 

\begin{equation*}
  \lim_{k\to\infty} \left(A^k\right)_{ij} = 0, \text{ for all } i,j
\end{equation*}

\ex Consider $A=\begin{bmatrix}
\frac{1}{2} & 0\\
16 & \frac{1}{2}\\
\end{bmatrix}$

\begin{align*}
  A^2 &= 
  \begin{bmatrix}
    \frac{1}{4} & 0 \\
    16 & \frac{1}{4}
  \end{bmatrix} \\
  A^3 &= 
  \begin{bmatrix}
    \frac{1}{8} & 0\\
    12 & \frac{1}{8}\\
  \end{bmatrix}\\
  A^4 &= 
  \begin{bmatrix}
    \frac{1}{16} & 0\\
    8 & \frac{1}{16}\\
  \end{bmatrix} \\
  A^k &= 
  \begin{bmatrix}
    \frac{1}{2^k} & 0\\
    P_k & \frac{1}{2^k}\\
  \end{bmatrix}
\end{align*}

where $P_k = \begin{cases}
16 & k=1\\
\frac{16}{2^{k-1}} + \frac{1}{2}P_{k-1} & k>1
.\end{cases}$

Since $\lim_{k\to\infty} P_k = 0$, we also know that $\lim_{k\to\infty} P_k = 0$.
$\therefore A$ is a convergent matrix.

Notice that this convergent matrix has a spectral radius (see Lecture 12 notes
, page 5) less than 1.

This generalizes:
\pagebreak

\thm The following statements are equivalent:

\begin{enumerate}
  \item $A$ is a convergent matrix.
  \item $\rho(A) < 1$
  \item $\lim_{n\to\infty} A^n x = 0$ for every $x$
  \item $\lim_{n\to\infty} \left|\left|A^n\right|\right| = 0$ for all natural
    norms $||\cdot||$
\end{enumerate}

Iterative techniques convert the system $Ax=b$ into an equivalent system of
the form $x=Tx+c$ where $T$ is a fixed matrix and $c$ is a vector. An initialv
vector $x^{(0)}$ is chosen, and then a sequence of approximate solution vectors
is generated:

\begin{equation*}
  x^{(k)} = Tx^{(k-1)} + c
\end{equation*}

Iterative techniques are rarely used in very small systems 
(i.e. when $n^3$ is small). In these cases, iterative techniques may be slower 
since they require several iterations to obtain the desired accuracy.

\uline{\textbf{IDEA:}} It is possible to \enquote{split} the matrix $A:$

\begin{align*}
  Ax&=b \\
  \left[M+(A-M\right]x &= b \\
  Mx &= b+(M-A)x \\ 
  x &= (I-M^{-1}A)x + M^{-1}b 
\end{align*}

Iteration becomes 

\begin{equation*}
  x^{(k+1)} = (I-M^{-1}A)x^{(k)} + M^{-1}b
\end{equation*}

We set $T\cong I-M^{-1}A$ (the amplification matrix) and $c\cong M^{-1}b$.

\begin{equation*}
  x^{(k+1)} = Tx^{(k)} + c
\end{equation*}

How do we choose $M$?

We want:

\begin{enumerate}
\item $M$ easy to \enquote{invert}
\item M \enquote{close to $A$} in the sense that $\rho(T)$ is small.
\end{enumerate}

\ex Let $M=D=\begin{bmatrix}
  a_{11} & 0 & 0\\
0 & \ddots & 0\\
0 & 0 & a_{nn}\\
\end{bmatrix}$

This gives the \uline{Jacobi Iterative Method}.

In the text's notation,

\begin{equation*}
  A=D-L-U
\end{equation*}

Where $D$ is diagonal, $L$ is lower triangular, and $U$ is upper triangular.

\begin{align*}
  Ax &= b \\
  (D-L-U)x &= b \\
  Dx &= (L+U)x + b\\
  x &= D^{-1}(L+U)x +D^{-1}b \\
\end{align*}

Which results in the iteration 

\begin{equation*}
  x^{(k+1}) = D^{-1}(L+U)x^{(k)} + D^{-1}b
\end{equation*}

Let $T=D^{-1}(L+U)$ and $c=D^{-1}b$.

\begin{equation*}
  x^{(k+1)} = Tx^{(k)} + c
\end{equation*}

See example in the notes, there are too many matrices to type out in LaTeX.
See \enquote{Chapter 7.pdf} page 27 (7-36.7)

% Click: \href{run:./Chapter 7.pdf#page=27}{Chapter 7.pdf} page 27 (7-36.7)

\subsubsection{Comments on Jacobi's Method}

\begin{equation*}
  x^{(k+1)} = D^{-1}(L+U)x^{(k)} + D^{-1}b
\end{equation*}

\begin{enumerate}
  \item The algorithm requirs $a_{ii} \neq 0$ for $i=1,\dots,n$ If one of the 
    $a_{ii} = 0$, and the system is nonsingular, then a reordering of the 
    equations can be performed so that no $a_{ii} = 0$.
  \item To speed convergence, the equations should be arranged such that
    $|a_{ii}$ is as large as possible.
  \item A possible stopping criterion is to iterate until
    $\frac{||x^{(k)}-x^{(k-1)}||}{||x^{(k-1)}||} \leq \epsilon$
\end{enumerate}

If we write out Jacobi's Method

\begin{equation*}
  x^{(k+1)} = D^{-1}(L+U)x^{(k)} + D^{-1}b
\end{equation*}

we find that

\begin{equation*}
  x_i^{(k+1)} = \frac{\sum_{j=1; j\ne i}^n (-a_{ij}x_j^{(k)}) +b_i}{a_{ii}}
\end{equation*}

Notice that to compute $x_i^{(k+1)}$, the components $x_i^{(k)}$ are used.
But, for $i>1$, $x_1^{(k+1}, x_2^{(k+1)}, \dots, x_n^{(k+1)}$ have already
been computed and are likely better approximations to the actual solutions than

\begin{equation*}
  x_1^{(k)}, x_2^{(k)}, \dots, x_n^{(k)}
\end{equation*}

So it seems reasonable to compute with these most recently computed values.

\ie:

\begin{equation*}
  x_i^{(k+1)} = \frac{-\sum_{j=1}^{i=1}(a_{ij}x_j^{(k+1)} - \sum_{j=i+1}^{n} (a_{ij} x_j^{(k)} +b_i}{a_{ii}}
\end{equation*}

This is called the Gauss-Seidel iterative technique, and it also has a matrix
formulation with $M\cong (D-L):$

\begin{align*}
  Ax &= b \\
  (D-L-U)x &= b \\
  (D-L)x &= Ux+b \\
  x &= (D-L)^{-1}Ux+(D-L)^{-1}b
\end{align*}

$\implies$ iteration becomes

\begin{equation*}
  x^{(k+1)} = (D-L)^{-1}Ux^{(k)} + (D-L)^{-1}b
\end{equation*}

*Notice that $D-L$ is lower triangular. It is invertible $\iff$ each $a_{ii}\ne 0$

