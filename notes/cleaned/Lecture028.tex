\section*{Lecture Outline}

\begin{enumerate}
\item Starts with a review of the previous lecture, including:
  \begin{enumerate}[label=(\alph*)]
  \item Interpolation with two points (Trapezoidal Rule)
  \item Interpolation with three points (Simpson's Rule)
  \end{enumerate}
\item The \textbf{error analysis} of numerical integration methods.
  \textbf{(\ref{sec:error_analysis})}.
  \begin{enumerate}[label=(\alph*)]
  \item Simpson's Rule, despite using a quadratic approximation, can integrate
    up to a cubic. (Integrating a quadratic gives a cubic, and the method
    specifically finds a nice cancellation that allows higher order
    approximations)
  \item Definition of the degree of accuracy \ref{sec:degree_of_accuracy}
  \end{enumerate}
\item It turns out that Simpson's Rule and the Trapezoidal Rule are not the only
  methods of numerical integration. In fact, there are infinitely many methods
  of numerical Integration: The \textbf{Newton-Cotes Formulas
  (\ref{sec:newton_cotes_formulas})}.
  \begin{enumerate}[label=(\alph*)]
  \item Even-degree Newton-Cotes formulas are more accurate than odd.
  \item The formulas are \textbf{exact} if the degree of accuracy
    is greater than or equal to the degree of the function it is approximating.
  \item The Newton-Cotes formulas are \textbf{closed} if the endpoints of the
    interval are included as nodes. The formulas are \textbf{open
    (\ref{sec:open_newton_cotes_formulas})} if the nodes are all contained in
    the open interval $(a,b)$.
  \end{enumerate}
\item Finally, we discuss the \textbf{composite numerical integration
  (\ref{sec:composite_numerical_integration})} method. This method uses
  Newton-Cotes formulas to approximate the integral of a function over an
  interval.
\end{enumerate}

\section{Error Analysis of Numerical Integration Methods}
\label{sec:error_analysis}

\begin{center}
  \begin{tabular}{c|c|c}
    $f(x)$ & Simpson's & Trapezoidal \\ 
    \hline
    $x$ & 0 & 0 \\
    $x^3$ & nonzero & 0
  \end{tabular}
\end{center}

Simpson's Rule can integrate up to a cubic, even though the function used to
approximate the integral is a quadratic.  This is because in Simpson's Rule,
the quadratic approximation is integrated and there is a cancellation that
occurs, which allows the quadratic approximation to be used to approximate
the integral.

\subsection{Degree of Accuracy}
\label{sec:degree_of_accuracy}

\defn The \textbf{degree of accuracy} or precision of a quadratic formula is the largest
positive integer $n$ such that the formula is exact for $x^k$ when $k=0,1,\dots,n$.

\begin{center}
  \begin{tabular}{c|c}
    & Degree of Accuracy \\ \hline
    Trapezoid Rule & 1 \\
    Simpson's Rule & 3 \\
  \end{tabular}
\end{center}

\section{Newton-Cotes Formulas}
\label{sec:newton_cotes_formulas}

The Trapezoid and Simpson's Rules are examples of \textbf{Newton-Cotes
Formulas}. The $(n+1)$-point \uline{closed} Newton-Cotes formula uses nodes $x_i
= x_0+ih$ for $i=0,1,\dots,n$ where 

\begin{align*}
  x_0 &= a, \\
  x_n &= b, \\
  h &= \frac{b-a}{n}
\end{align*}

\noindent
Then,

\begin{align*}
  \int_{a}^{b} f(x) \, dx \approx \int_{a}^{b} P_n(x) \, dx 
  &= \int_{a}^{b} \sum_{i=0}^{n} h_i(x)f(x_i) \, dx \\
  &= \sum_{i = 0}^{n} \int_{a}^{b} L_i(x)f(x_i) \, dx \\
  &= \sum_{i = 0}^{n} a_i f(x_i)
\end{align*}

where $\displaystyle a_i = \int_{a}^{b} L_i(x) \, dx$

The formula is closed because the endpoints of the interval are included as
nodes. An error analysis of the Newton-Cotes formulas gives an interesting
result:

\thm Suppose that $\displaystyle \sum_{i = 0}^{n} a_if(x_i)$ denotes the $n+1$
point closed Newton-Cotes formula with $x_0=a, x_n=b$ and $h = \frac{b-a}{n}$.
If $n$ is even and $f\in C^{n+2}[a,b]$ then there exists $\xi \in (a,b)$ with 

\[
\int_{a}^{b} f(x) \, dx = \sum_{i = 0}^{n} a_if(x_i) +
\frac{h^{n+3}f^{(n+2)}(\xi)}{(n+2)!} \int_{0}^{n} t^2(t-1)\dots (t-n) \, dt
.\]

If $n$ is odd and $f\in C^{n+1}[a,b]$ then there exists $\xi \in (a,b)$ with

\[
  \int_{a}^{b} f(x) \, dx = \sum_{i = 0}^{n} a_if(x_i) +
  \frac{h^{n+2}f^{n+1}(\xi)}{(n+1)!} \int_{0}^{n} t(t-1)\dots(t-n) \, dt
.\]

Notice that the degree of precision is $n+1$ and the error is $O(h^{n+3})$ if
$n$ is even. If $n$ is odd then the degree of precision is only $n$ and hte
error is only $O(h^{n+2})$.

\begin{center}
  \begin{tabular}{c|l c}
    $n$ & \textbf{Name} & \textbf{Error Term} \\
    \hline
    1 & Trapezoid Rule & $-\frac{h^3}{12} f''(\xi)$ \\
    2 & Simpson's Rule & $-\frac{h^5}{90} f^{(4)}(\xi)$ \\
    3 & Simpson's Three-Eighths Rule & $-\frac{3h^5}{80} f^{(4)}(\xi)$ \\
    4 & & $-\frac{h^7}{945} f^{(6)}(\xi)$ \\
  \end{tabular}
\end{center}

\subsection{Open Newton-Cotes Formulas}
\label{sec:open_newton_cotes_formulas}
There are also \uline{open} Newton-Cotes formulas. here,

\begin{align*}
x_i &= x_0+ih \qquad i = 0,1,\dots,n \\
x_0 &= a+h \\
h = \frac{b-a}{n+2}
\end{align*}

Then the open Newton-Cotes formulas are given by

\begin{equation}
  \int_{a}^{b} f(x) \, dx \approx \sum_{i = 0}^{n} a_if(x_i) 
.\end{equation}

where $\displaystyle a_i = \int_{a}^{b} L_i(x) \, dx$

Note that $x_0 = a+h$ and $x_n = b-h$. The formulas are \uline{open} because the
nodes are all contained in the open interval $(a,b)$. Once again, if $n$ is even,
the degree of precision is $(n+1)$ and the error is $O(h^{n+3})$. If $n$ is odd,
the degree of precision is only $n$ and the error is only $O(h^{n+2})$.

Some examples of open Newton-Cotes formulas are:

\begin{tabular}{c| l}
    $n$ & \\
    \hline
    0 & $\displaystyle 2h f(x_0) + \frac{h^3}{3} f''(\xi), \quad \text{where } \xi \in (a,b)$ \\[10pt]
    1 & $\displaystyle \frac{3h}{2} \left[ f(x_0) + f(x_1) \right] + \frac{3h^3}{4} f''(\xi), \quad \text{where } \xi \in (a,b)$ \\[10pt]
    2 & $\displaystyle \frac{4h}{3} \left[ 2f(x_0) - f(x_1) + 2f(x_2) \right] + \frac{14h^5}{45} f^{(4)}(\xi)$ \\[10pt]
    3 & $\displaystyle \frac{5h}{24} \left[ 11f(x_0) + f(x_1) + f(x_2) + 11f(x_3) \right] + \frac{95}{144} h^5 f^{(4)}(\xi)$ \\
\end{tabular}

$n=0$ is also called the \textbf{Midpoint Rule}. 

\section{Composite Numerical Integration}
\label{sec:composite_numerical_integration}
Typically, we do not apply Newton-Cotes formulas directly onto the interval
$[a,b]$. If we did, then high degree formulas would be required to obtain
accurate solutions, but as we have already seen, these high degree polynomials
would give an oscillatory and innacurate interpolation. To avoid this problem,
we prefer a piecewise approach to numerical integration that uses low order
Newton-Cotes formulas.

\Ex Simpson's Rule

\begin{center}
  \begin{tikzpicture}[scale=2.0]
    % Draw axes
    \draw[->] (-0.1,0) -- (6.5,0) node[below] {$x$};
    \draw[->] (0,-0.1) -- (0,2.2) node[left] {$y$};

    % Define function
    \draw[thick, red, domain=0:6, samples=100] 
      plot(\x, {0.05375*\x*\x*\x - 0.520625*\x*\x + 1.4803125*\x});

    % Draw dotted vertical lines
    \foreach \x in {0.5,1,1.5,2,2.5,4.5,5,5.5} {
      \draw[thick, dotted, blue] (\x,0) -- (\x,{0.05375*\x*\x*\x - 0.520625*\x*\x + 1.4803125*\x});
    }

    \foreach \x in {0.5, 1.5, 2.5, 4.5, 5.5} {
      \draw[thick, blue] (\x,0) -- (\x,{0.05375*\x*\x*\x - 0.520625*\x*\x + 1.4803125*\x});
      % \node[below] at (\x,-0.05) {$x_{\x}$};
    }

    % \node at (3.5, 0.5) {$\LARGE{\dots}$};
    \foreach \i in {0,1,2} {
        \fill (3.3 + \i*0.2, 0.5) circle (1.0pt);
    }

    \node[below] at (0.5, -0.05) {$x_0$};
    \node[below] at (1.5, -0.05) {$x_2$};
    \node[below] at (2.5, -0.05) {$x_4$};
    \node[below] at (4.5, -0.05) {$x_{n-2}$};
    \node[below] at (5.5, -0.05) {$b=x_n$};

  \end{tikzpicture}
\end{center}

% image goes here

We divide the interval into an even number of subintervals. Simpson's rule is
applied on each consecutive pair of subintervals.

\[
  \frac{h}{3} \left[f(x_i) + 4f(\frac{x_i+x_{i+2}}{2}) + f(x_{i+2})\right]
.\]

Take $\displaystyle h=\frac{(b-1)}{n}$ and $x_j = a + jh$. Then,

\begin{align*}
  \int_{a}^{b} f(x) \, dx &= \sum_{j=1}^{\frac{n}{2}} \int_{x_{2j-2}}^{x_{2j}}
  f(x) \, dx \\
                          &= \sum_{j=1}^{\frac{n}{2}} \left\{
                            \frac{h}{3} \left[
                              f(x_{2j-2}) + 4f(x_{2j-1}) + f(x_{2j})
                            \right] - \frac{h^5}{90} f^{(4)}(\xi_j)
                          \right\} \\
                          &\qquad\text{where } x_{2j-2} < \xi_j < x_{2j} \\
                          &\qquad\text{and } f \in C^4[a,b]
\end{align*}

taking into account that $f(x_{2j}), 0 < j < \frac{n}{2}$ appears in 2 terms,
this summation can be simplified somewhat

\[
  \int_{a}^{b} f(x) \, dx = \frac{h}{3} \left[
    f(x_0) + 2 \sum_{j=1}^{\frac{n}{2}-1} f(x_{2j}) + 4 \sum_{j=1}^{\frac{n}{2}}
    f(x_{2j-1}) + f(x_n)
  \right] + \text{error}
.\]


got to page 41/80 (32/54 in the published lecture slides)
