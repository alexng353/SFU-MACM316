\section*{Lecture Outline}
\begin{enumerate}
\item 
\end{enumerate}

\section{Error Behavior of Simpson's Rule (19.12)}

It is important to understand the stability property of \textbf{Composite
Newton-Cotes} integration techniques.

Assume $f(x_i)$ is approximated by $\tilde{f}(x_i)$:

\begin{equation*}
  f(x_i) = \tilde{f} (x_i) + e_i \qquad 0 \leq i \leq n
.\end{equation*}

\noindent
where $e_i$ is the roundoff associated with using $\tilde{f}$ to approximate
$f$. Then, the accumulated roundoff error in the Composite Simpson's Rule is

\begin{align*}
  |e(h)| &= \left|\frac{h}{3} \left[
    e_0 + 2 \sum_{j=1}^{\frac{n}{2}-1} e_{2j} + 4 \sum_{j=1}^{\frac{n}{2}}
    e_{2j-1} + e_n
  \right]\right| \\
         & \leq \frac{h}{3} \left[
           |e_0| + 2 \sum_{j=1}^{\frac{n}{2}-1} |e_{2j}| + 4 \sum_{j=1}^{\frac{n}{2}}
           |e_{2j-1}| + |e_n|
         \right]
\end{align*}

We have a triangle inequality. Assume all $e_j$ are bounded by $\bigEps$.

\begin{equation*}
  h = \frac{(b-a)}{n}
\end{equation*}


\begin{align*}
  |e(h)| & \leq \frac{h}{3} \left[
    \bigEps + 2(\frac{n}{2} -1)\bigEps + 4 \frac{n}{2} \bigEps + \bigEps
  \right] \\
         &= \frac{h}{3} 3n \bigEps \\
         &= nh \bigEps \\
         &= (b-a) \bigEps \\
\end{align*}

Which is indepentent of $h$ which implies the procedure is stable as $h\to 0$

\section{Romberg Integration}
\label{sec:romberg_integration}
An interesting point concerning the composite Trapezoid Rule:

If $f= \in C^2[a,b]$ then there exists a $\mu \in [a,b]$ such that 

\begin{align*}
  \int_{a}^{b} f(x) \, dx &= \frac{h}{2} \left[
    f(a) + 2 \sum_{j=1}^{h-1} f(x_j) + f(b)
  \right] - \frac{b-a}{12}h^2 f''(\mu) \\
                          &\text{where } h = \frac{b-1}{n} \\
                          &\text{and } x_j = a + jh
\end{align*}

Thus the error for the composite Trapezoid Rule is $O(h^2)$. In fact we can be
more precise. An application of the Euler-MacLaurin summation formula shows that
for sufficiently smooth $f$,

\begin{align*}
  \text{error} \quad&\quad= c_1 h^2 + c_2 h^4 + \dots + c_m h^{2m} + O(h^{2m+2}) \\
  \text{where}\quad &c_k = \text{const} \times \left(
    f^{(2k-1)}(b) - f^{(2k-1)}(a)
  \right)
\end{align*}

This shows us that the Composite Trapezoid Rule is extremely accurate for smooth
periodic functions, provided $h$ is small enough.

*Note: The error expansion contains only even powers of $h$, so eliminating the 
leading error term improves the accuracy by two additional orders of $h$.

Notice that we know the form of the error, so we can obtain higher order
accuracy by using Richardson Extrapolation. (To give Romberg Integration)

\subsection{Richardson Extrapolation to obtain Romberg Integration}
\label{sec:richardson_extrapolation_to_obtain_romberg_integration}

We will carry out Composite Trapezoid Rule approximations with 

\[
  m_1 = 1, m_2 = 2, m_3 = 4, \dots, m_n = 2^{n-1} \text{ intervals.} 
.\]

The values of the step sizes $h_k$ corresponding to $m_k$ are 

\[
  h_k = \frac{(b-1)}{m_k} = \frac{(b-1)}{2^{k-1}}
.\]

With this notation, the Composite Trapezoid Rule becomes

\begin{equation*}
  \int_{a}^{b} f(x) \, dx = \frac{h_k}{2} \left[
    f(a) + 2\left(
      \sum_{i=1}^{2^{k-1}-1} f(a+ih_k) 
      \right)
      - \frac{(b-a)}{12} h^2_k f''(\mu_k)
  \right]
.\end{equation*}

where $\mu_k \in (a,b)$

Let $R_{k,1}$ be the approximation to the integral using $m_k=2^{k-1}$
intervals.

\ie 

\begin{align*}
  R_{1,1} &=  \frac{h_1}{2}\left[
    f(a)+f(b)
  \right] = \frac{(b-1)}{2}
  \left[
    f(a)+f(b)
  \right]
\end{align*}

\begin{align*}
  R_{2,1} &= \frac{h_2}{2} \left[
    f(a)+f(b)+2f(a+h_2)
  \right] \\
          &= \frac{(b-a)}{4} \left[
            f(a)+f(b)+2f(a+\frac{b-1}{2})
          \right] \\
          &= \frac{1}{2} \left[
            R_{1,1} + h_1f(a+h_2)
          \right]
\end{align*}

notice that when $h$ is halved, all the old points at which the function was
evaluated appear in the new computation- we can avoid repeating the evaluations.

\begin{equation*}
  R_{3,1} = \frac{1}{2} \left\{
    R_{2,1} + h_2 \left[
      f(a+h_3) + f(a+3h_3)
    \right]
  \right\}
.\end{equation*} 

\[
\vdots
\]

\begin{equation*}
  R_{k,1} = \frac{1}{2} \left\{
    R_{k-1,1} + h_{k-1} \left[
      \sum_{i=1}^{2^{k-2}} f(a+(2i-1)h_k)
    \right]
  \right\}
.\end{equation*}

We can apply this equation to perform the first step of Romberg Integration for

\begin{equation*}
  \int_{0}^{1} e^{-x} \, dx = 1-e^{-1} \approx 0.63212
.\end{equation*}

\begin{align*}
R_{1,1} &= \frac{(1-0)}{2} [e^{-0} + e^{-1}] \approx 0.68394 \\
R_{2,1} &= \frac{1}{2}\left[R_{1,1} + \frac{(1-0)}{2}e^{-(0+\frac{1}{2})}
\right] = 0.64523 \\
R_{3,1} &= 0.65341 \\
R_{4,1} &= 0.63294 \\
\end{align*}

We can obtain a faster convergence using Richardson Extrapolation:

Notice that:

\begin{align*}
\int_a^b f(x)\,dx - R_{h,1} 
&= \sum_{i=1}^{m} c_i h^{{2i}} + O(h^{2m+2}) \\
&= c_1 h^2 + \sum_{i=2}^{m} c_i h^{2i} + O(h^{2m+2})
\end{align*}

\begin{align*}
\int_a^b f(x)\,dx - R_{h/2,1} 
&= \sum_{i=1}^{m} c_i h_{h/2}^{2i} + O(h^{2m+2}) \\
&= \frac{c_1 h^2}{4} + \sum_{i=2}^{m} \left(\frac{c_i h^{2i}}{4^i}\right) + O(h^{2m+2})
\end{align*}

Subtracting the first from 4 times the second gives an $O(h_k^4)$ formula:

\begin{align*}
  \int_a^b f(x)\,dx - R_{h,2} 
&= \sum_{i=2}^{m} \frac{c_i}{3} \left( \left( \frac{h^{{2i}}}{4^i} \right) -
h^{2i} \right) + O(h^{2m+2}) \\
  \text{where} \quad R_{h,2} &= \frac{R_{h,1} + R_{h/2,1} - R_{h,1}}{3}
\end{align*}

Of course, this procedure can be repeated to eliminate the $O(h_k^4)$ term from
the error. Continuing in this manner, we have an $O(h_k^{2j})$ approximation
formula defined by 

\begin{equation*}
  R_{kj} = R_{k, j-1} + \frac{R_{k,j-1} - R_{k-1, j-1}}{4^{j-1}-1}
.\end{equation*}

\Ex Use Romberg Integration to approximate 

\[
  \int_{0}^{1} e^{-x} \, dx
.\]

to $5$ significant digits.

\begin{tabular}{c|ccccc}
      & $R_{h,1}$ & $R_{h,2}$ & $R_{h,3}$ & $R_{h,4}$ & $R_{h,5}$ \\
\hline
  $R_{1,j}$ & .6839397 &          &          &        &            \\
  $R_{2,j}$ & .6452352 & .6723337 &          &        &            \\
  $R_{3,j}$ & .6354094 & .6321312 & .6321209 &        &            \\
  $R_{4,j}$ & .6329434 & .6321214 & .6321206 & .6321206 &          \\
  $R_{5,j}$ & .6323263 & .6321206 & .6321206 & .6321206 & .6321206 \\
\end{tabular}

A typical stopping criterion is that both

\[
  |R_{n-1, n-1} - R_{n,n} \text{ and } |R_{n-2,n-2} - R_{n,n}| < \bigEps
.\]

for some error tolerance $\bigEps$.

Note that we may not observe the expected convergence acceleration if

\begin{itemize}
  \item The integrand $f$ is not sufficiently smooth. We need $f\in
    C^{2k+2}[a,b]$ to generate the $k^{th}$ row of the table.
  \item The coefficients $c_1, c_2, \dots$ are very small. This happens for
    periodic functions if the interval of integration is an integer multiple of
    the period or for functions with extremely small derivatives at the
    endpoints of the interval of integration.
\end{itemize}

\section{Adaptive Quadrature}

Composative quadrature rules necessitate the use of equally spaced points. This
does not take into account that some portions of the curve may have large
functional variations that require more attention than other portions of the
curve. It is useful to introduce a method that adjusts the step size to be
smaller over portions of the curve where a larger functional variation occurs.
Thi technique is called adaptive quadrature. We will now discuss an adaptive
based on Simpson's Rule. The other composite procedures can be modified in a
similar manner.

