% \section{Continued from Lecture 18}

Newton's Method can only be applied to simple zeros of a function.
Identification fo the multiplicity of a zero is often made easier by the two
following theorems.

\thm 2.10

$f\in C'[a,b]$ has a simple zero at $p$ in $(a,b)$ if and only if $f(p) = 0$
but $f'(p) \neq 0$.

\thm 2.11

The function $f\in C^m[a,b]$ has a zero of multiplicity $m$ at $p$ if and only
if 

\[
  0 = f(p) = f'(p) = f''(p) = \dots = f^{(m-1)}(p)
.\]

but $f^{(m)}(p) \neq 0$.

We want to obtain quadratic convergence with Newton's Method for multiple roots.

One approach is to define a new function

\[
  \mu(x) = f\frac{(x)}{f'(x)}
.\]

We assume $p$ is a zero of multiplicity $m$ and $f(x) = (x-p)^m q(x)$ where
$q(p) \neq 0$. Then,

\begin{align*}
  \mu(x) &= \frac{(x-p)^m}{m(x-p)^{m-1} q(x) + q'(x)(x-p)^m} \\
  &= \frac{(x-p)q(x)}{mq(x) + q'(x) (x-p)}  \\
  &= (x-p) \frac{q(x)}{mq(x)+q'(x-p)} \\
  q(p) \neq 0 \quad\text{therefore, } &\mu(p) \text{ has a simple root at } x=p
\end{align*}

so $\mu(p) = 0$, but $\displaystyle \frac{q(p)}{mq(p) + q'(p)(p-p)} = \frac{1}{m} \neq 0$.
and $p$ is a zero of multiplicity $1$ of $\mu(x)$.

\textbf{Good:}
\begin{itemize}
  \item Quadratic convergence for all roots
\end{itemize}

\textbf{Bad:}
\begin{itemize}
  \item Need $f''$
  \item $\mu$ is more expensive to work with
  \item $\mu$ might give more roundoff error
\end{itemize}

We return to finding the roots of

\[
p^3 - p^3 - p + 1 = 0
.\]

Apply:

\begin{table}[h]
    \centering
    \begin{tabular}{c c c}
        \toprule
         & Newton's Method & Modified Newton's Method \\
        \midrule
        $p_0$ & 1.1 & 1.1 \\
        $p_1$ & 1.05116... & 0.997735... \\
        $p_2$ & 1.02589... & 0.999999... \\
        $p_3$ & 1.01303... &  \\
        $p_4$ & 1.00653... &  \\
        $p_5$ & 1.00327... &  \\
        \bottomrule
    \end{tabular}
    % \caption{Comparison of Newton's Method and Modified Newton's Method}
    % \label{tab:newton_methods}
\end{table}

This table shows that Newton's Method has linear convergence, and the Modified
Newton's Method has quadratic convergence.

\section{Congervence Speed and Acceleration}

Suppose that we are given a linearly convergent sequence, and that we want to
speed up the convergence. We want to analyze the behaviour of the error and use
this knowledge to greatly reduce the error. 

For example, we saw last day that the iterator method with 

\begin{align*}
  p_{n+1} &= g(p_n) \\
  \text{and } g(x) &= \frac{1}{2}\sqrt{10-x^3}
\end{align*}

gives only linear convergence and the limit $p=1.3652$

\begin{table}[h]
    \centering
    \begin{tabular}{c c}
        \toprule
        Iteration & $p - p_n$ \\
        \midrule
        $p - p_0$ & $-0.13476\ldots$ \\
        $p - p_1$ & $+0.07827\ldots$ \\
        $p - p_2$ & $-0.03710\ldots$ \\
        $p - p_3$ & $+0.01977\ldots$ \\
        $p - p_4$ & $-0.00994\ldots$ \\
        \bottomrule
    \end{tabular}
    % \caption{Error progression in iterations}
    % \label{tab:error_progression}
\end{table}

Notice that the \uline{ratio} of the errors is fairly constant- we can use this
idea to accelerate the convergence of the method.

Suppose $\displaystyle \frac{p_{n+1}-p}{p_n-p} \approx \frac{p_{n+2}-p}{p_{n+1}-p}$

\[
  \therefore p \approx \frac{p_n p_{n+1} - p_{n+1}^2}{p_{n+2}-2p_{n+1}+p_n}
.\]

or,

\[
  p \approx p_n - \frac{p_{n+1}^2 - 2p_n p_{n+1} + p_n^2}{p_{n+2}-2p_{n+1}+p_n}
.\]

which is derived by adding and subtracting $p_n$ to the right hand side.

The corresponding sequence

\[
  \hat{p}_{n+1} = p_n - \left[ \frac{p_{n+1}^2-2p_np_{n+1}+p_n^2}{p_{n+2}-2p_{n+1}+p_n} \right]
.\]

is known as Aitken's Method.

\pagebreak
Applying Aitken's method to our previous sequence,

\begin{table}[h]
    \centering
    \begin{tabular}{c c | c c}
        \toprule
        \multicolumn{2}{c|}{Fixed Point Iteration} & \multicolumn{2}{c}{Aitken's Method} \\
        \midrule
        Iteration & $p - p_n$ & Iteration & $\hat{p} - p_n$ \\
        \midrule
        $p - p_0$ & $-0.13476\ldots$ & $\hat{p} - p_0$ & $0.00090088$ \\
        $p - p_1$ & $+0.07827\ldots$ & $\hat{p} - p_1$ & $0.00023088$ \\
        $p - p_2$ & $-0.03731\ldots$ & $\hat{p} - p_2$ & $0.00006107$ \\
        $p - p_3$ & $+0.01977\ldots$ & $\hat{p} - p_3$ & $0.00001592$ \\
        $p - p_4$ & $-0.00994\ldots$ & $\hat{p} - p_4$ & $0.00000418$ \\
        \bottomrule
    \end{tabular}
    % \caption{Error progression in Fixed Point Iteration and Aitken's Method}
    % \label{tab:fixed_aitken}
\end{table}

we notice that Aitken's Method is much faster.
It can be shown that the following theorem holds:

\section{Theorem (2.13 of Text)}

Suppose that $\{ p_n \}$ is a sequence that converges linearly to the limit $p$
and that for all sufficiently large values of $n$ we have $(p_n-p)(p_{n+1}-p) > 0$.

Then the sequence $\{ \hat{p}_{n=0}^\infty \}$ converges faster than $\{ p_n \}_{n=0}^\infty$.
to $p$ in the sense that $\displaystyle \lim_{n\to\infty} \frac{\hat{p}_n - p}{p_n-p} = 0$

The theorem does not apply to alternating sequences, but as we saw from our 
example it is often very useful even there for accelerating convergence.

Finally, we remark that this method is often written using difference operations 
to simplify the notation.

... sorry i missed a bunch here

\section{Zeros of Polynomials}

We want to compute the zeros (roots) of polynomials. A polynomial of degree $n$
has the form

\[
  P(x) = a_nx^n + a_{n-1}x^{n-1} + \dots + a_1x+a_0
.\]

where $a_i , i=0,\dots,n$ are constants and $a_n \neq 0$.

\subsection{Fundamental Theorem of Algebra}

If \( P \) is a polynomial of degree \( n \geq 1 \), then \( P(x) \) has at least one (possibly complex) root.

\subsubsection*{Corollary:}
If \( P(x) \) is a polynomial of degree \( n \geq 1 \), then there exist unique constants \( x_1, x_2, \dots, x_k \) (possibly complex), and positive integers \( m_1, m_2, \dots, m_k \) such that

\[
  \sum_{i=1}^{k} m_i = n
\]

and 

\[
  P(x) = a_n (x - x_1)^{m_1} (x - x_2)^{m_2} \dots (x - x_k)^{m_k}.
\]

\subsubsection*{Corollary:}
Let \( P \) and \( Q \) be polynomials of degree at most \( n \). If \( x_1, x_2, \dots, x_{n+1} \) with \( n+1 > n \) are distinct numbers with 

\[
  P(x_i) = Q(x_i) \quad \text{for } i = 1,2, \dots, n+1,
\]

then 

\[
  P(x) = Q(x) \quad \text{for all } x.
\]

We want to use Newton's Method to locate the approximate roots of $P$. It will
be necessary to evaluate $P$ and its derivative at specified values. We now 
direct our attention to efficient methods for this task. The idea is to use 
nesting to evaluate an arbitrary $n^{th}$ degree polynomial using only

\[
  n \text{ multiplications and } n \text{ additions}
.\]

For illustration, consider $n=4$. To evaluate 
$P(x) = a_4x^4 + a_3x^3 + a_2x^2 + a_1x+a_0$, we write:

\[
  % P(x) = \left(\left(\left(\left(a_4x+a_3\right)x+a_2\right)x+a_1\right)x+a_0
    P(x) = \pqty{
      \pqty{
        \pqty{
          a_4x + a_3
        } x + a_2
      } x + a_1
    } x + a_0
.\]

Algorithmically,

\begin{enumerate}
  \item set $b_4 = a_4$
  \item set $b_3 = b_4x + a_3$
  \item set $b_2 = b_3x + a_2$
  \item set $b_1 = b_2x + a_1$
  \item set $b_0 = b_1x + a_0$
\end{enumerate}

Now $b_0$ gives the value $P(x)$.

For general polynomials of degree $n:$

\begin{align*}
  b_n &= a_n \\
  b_k &= a_k+b_{k+1}x \qquad\qquad 0 \leq k \leq n-1 \\
  \text{and } b_0 &= P(x) \\
\end{align*}

we also want $P'(x)$. Differentiate each of the steps listed above, keeping in
mind that $b_i$ is a function of $x$. We get:

\begin{align*}
  b_n ' &= 0 \\
  b_k' &= b_{k+1}'x + b_{k+1} \\
  \text{and } b_0' &= P'(x) \\
\end{align*}

Relabel: $c_{n+1} = b_n'$. Then an efficient method for computing $P'(x)$ is

\begin{align*}
  c_n &= a_n \\
  c_k &=  c_{k+1}x + b_{k} \\
  \text{and then } c_1 &= P'(x) \\
\end{align*}

This is called Horner's Method for evaluating a polynomial.

\section{Horner's Method}

Honer's Method has another useful property. 

Consider $P(x)$

\begin{align*}
    &= a_n x^n + a_{n-1} x^{n-1} + \dots + a_1 x + a_0 \\
    &= b_n x^n + (b_{n-1} - b_n x_0) x^{n-1} + \dots + (b_1 - b_2 x_0) x + (b_0 - b_1 x_0) \\
    &= (b_n x^{n-1} + b_{n-1} x^{n-2} + \dots + b_2 x + b_1) x \\
    &\qquad - (b_n x^{n-1} + b_{n-1} x^{n-2} + \dots + b_2 x + b_1) x_0 \\
    &\qquad + b_0 \\
    &= (b_n x^{n-1} + b_{n-1} x^{n-2} + \dots + b_2 x + b_1) (x - x_0) + b_0
\end{align*}

This property is useful because it gives us a way to find the next approximate
zero after we have found our first zeros.

\ie If $x_0$ is a root $P(x_0) = b_0 = 0\implies Q(x) (x-x_0) = P(x)$

so we can find the next root by examining roots of $Q(x)$.

Suppose we want additional roots of $P$. If $x_0$ is an approximate root of $P$,

\[
  P(x) \approx Q(x) (x-x_0) 
.\]

and the next step is toa pply Newton's Method to 

\[
  \displaystyle Q(X) = \frac{P(x)}{(x-x_0)}
.\]

this process is called deflation.

