\documentclass[12pt]{article}
\usepackage{amsmath}
\usepackage{amssymb}
\usepackage{amsthm}
\usepackage{amsfonts}
\usepackage{graphicx}
\usepackage{textcomp}
\usepackage{hyperref}
\usepackage{tikz}
\usepackage{enumitem}
\usepackage{mathtools}
\usepackage{enumitem}
\usepackage{wasysym}
\usepackage{ulem}
\usepackage{xspace}
\usepackage{booktabs}
\usepackage{physics}

\ifPDFTeX % ensure generation of machine readable output
\input{glyphtounicode}
\pdfgentounicode=1
\usepackage[T1]{fontenc}
\usepackage[utf8]{inputenc}
\usepackage{lmodern}
\fi

\usepackage{csquotes}

\DeclareMathOperator{\dist}{dist}
\DeclareMathOperator{\Nul}{Nul}
\DeclareMathOperator{\Row}{Row}
\DeclareMathOperator{\proj}{proj}

\setlength{\arraycolsep}{12pt}

\newcommand{\Eg}{\textbf{Eg.}\xspace}
\newcommand{\Ex}{\textbf{Ex.}\xspace}
\newcommand{\Ie}{\textbf{I.e.}\xspace}
\newcommand{\bigEps}{\mathcal{E}}
\newcommand{\bproof}{\textit{Proof ($\impliedby$).}\xspace}
\newcommand{\defn}{\textbf{Def.}\xspace}
\newcommand{\eg}{\textbf{e.g.}\xspace}
\newcommand{\ex}{\textbf{ex.}\xspace}
\newcommand{\fproof}{\textit{Proof ($\implies$).}\xspace}
\newcommand{\ie}{\textbf{i.e.}\xspace}
\newcommand{\lemma}{\textit{Lemma}\xspace}
\newcommand{\soln}{\textit{Soln.}\xspace}
\newcommand{\thm}{\textbf{Thm.}\xspace}

\renewcommand{\arraystretch}{1.25} % Adjust row spacing

\hypersetup{
    colorlinks=true,
    linkcolor=blue,
    filecolor=blue,      
    urlcolor=blue,
}

\newcommand{\ulhref}[2]{\href{#1}{\color{blue}\uline{#2}}}

\begin{document}

\title{MACM 316 Final Review}
\author{Alexander Ng}
\date{Monday, April 7, 2025}

\maketitle

\section{Example 1}
The following sequence converge to 0 of order $\alpha$. What is $\alpha$?
Identify any examples with sublinear convergence. Justify your answers using the
definitions
\begin{enumerate}[label=(\alph*)]
  \item $p_n = n^{-e}$
  \item $p_n = e^{-n}$
  \item $p_n=e^{-(e^n)}$
\end{enumerate}

\subsection{a)}
Conjecture $\alpha = 1$

\begin{align*}
  \lim_{n \to \infty} \frac{\abs{p_{n+1}-0}}{\abs{p_n-0}} &= \lim_{n\to\infty}
  \frac{(n+1)^{-e}}{n^{-e}} \\
                                                          &=
                                                          \lim_{n-?\infty}\bqty{1+\frac{1}{n}}^{-e} \\
                                                          &= 1 
\end{align*}

$\therefore \alpha = 1$. This is less than linear (sublinear) because $\lambda =
1$

\subsection{c)}

\begin{align*}
  \lim_{n\to\infty} \frac{\abs{p_{n+1}-0}}{\abs{p_n-0}} 
  &= \lim_{n\to\infty}
  \frac{e^{-(e^{n+1})}}{(e^{-(e^n)})^\alpha} \\
  &= \lim_{n\to\infty}\frac{e^{-e^n\cdot e}}{e^{-e^n\cdot \alpha}} \\
  &= 1 \text{ if } \alpha = e 
\end{align*}


\end{document}
