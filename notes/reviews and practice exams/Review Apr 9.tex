\documentclass[12pt]{article}
\usepackage{amsmath}
\usepackage{amssymb}
\usepackage{amsthm}
\usepackage{amsfonts}
\usepackage{graphicx}
\usepackage{textcomp}
\usepackage{hyperref}
\usepackage{tikz}
\usepackage{enumitem}
\usepackage{mathtools}
\usepackage{enumitem}
\usepackage{wasysym}
\usepackage{ulem}
\usepackage{xspace}
\usepackage{booktabs}
\usepackage{physics}

\ifPDFTeX % ensure generation of machine readable output
\input{glyphtounicode}
\pdfgentounicode=1
\usepackage[T1]{fontenc}
\usepackage[utf8]{inputenc}
\usepackage{lmodern}
\fi

\usepackage{csquotes}

\DeclareMathOperator{\dist}{dist}
\DeclareMathOperator{\Nul}{Nul}
\DeclareMathOperator{\Row}{Row}
\DeclareMathOperator{\proj}{proj}

\setlength{\arraycolsep}{12pt}

\newcommand{\Eg}{\textbf{Eg.}\xspace}
\newcommand{\Ex}{\textbf{Ex.}\xspace}
\newcommand{\Ie}{\textbf{I.e.}\xspace}
\newcommand{\bigEps}{\mathcal{E}}
\newcommand{\bproof}{\textit{Proof ($\impliedby$).}\xspace}
\newcommand{\defn}{\textbf{Def.}\xspace}
\newcommand{\eg}{\textbf{e.g.}\xspace}
\newcommand{\ex}{\textbf{ex.}\xspace}
\newcommand{\fproof}{\textit{Proof ($\implies$).}\xspace}
\newcommand{\ie}{\textbf{i.e.}\xspace}
\newcommand{\lemma}{\textit{Lemma}\xspace}
\newcommand{\soln}{\textit{Soln.}\xspace}
\newcommand{\thm}{\textbf{Thm.}\xspace}

\renewcommand{\arraystretch}{1.25} % Adjust row spacing

\hypersetup{
    colorlinks=true,
    linkcolor=blue,
    filecolor=blue,      
    urlcolor=blue,
}

\newcommand{\ulhref}[2]{\href{#1}{\color{blue}\uline{#2}}}

\begin{document}

\title{MACM 316 Review Apr 9}
\author{Alexander Ng}
\date{Wednesday, April 9, 2025}

\maketitle

\section{Problem 1}
Let $P(x)$ be the Lagrange interpolating polynomial for the function $f(x)$
and i missed the rest oops

\begin{align*}
  f(x) &= x^4 \\
  f'(x) &= 4x^3 \\
  f''(x) &= 12x^2 \\
  f^{(3)}(x) &= 24x
\end{align*}

*Write down the error form on your cheat sheet.*

\section{Problem 2}

Suppose $x_j = j$ for $j=0,1,2,3,4$, and $P_{0,1,2,3,4}(2) = a$, $P_{0,1,3,4}(2)
= b$ and $P_{1,3,4}(2) = c$. Find $P_{1,2}(2)$. If there is not enough data to
complete the question, or the data contains inconsistencies, explain why.

\soln This is the hard way:
\[
  \left.\underbrace{P_{0,1,2,3,4}(2)}_{a}\right|_{x=2} = 
    \left.
      \frac{
        (x-x_0)P_{1,2,3,4}(x) - (x-x_2) 
        \underbrace{
          P_{0,1,3,4}(x)
        }_{b}
      }{x_2-x_0}
      \right|_{x=2}
.\]
Using this information, we can theoretically find $P_{1,2,3,4}(2)$.
\textbf{However,} we can save ourselves lots of work by using the fact that we
already know that one of the points the interpolating polynomial passes through
is $x=2$.
\[
  P_{1,2}(2) = f(2)
.\]

\[
  P_{0,1,2,3,4}(2) = f(2) = a
.\]

\[
  \therefore P_{1,2}(2) = f(2) = a
.\]
\textbf{***The Lagrange Interpolating Polynomial for $f$ over the point $a$,
evaluated at $x=a$ is $P_{a}(a) = f(a)$.}

\section{Problem 3}

Suppose $f(x) = a+2x+3x^2+4x^3+5x^4$ and $x_i = i+1, i=0,1,2,3,4$. What is the
fourth divided difference $f\bqty{x_0, x_1, x_2, x_3, x_4}$?

The straightforward way (the hard way) to solve this problem is to setup the
divided difference table and evaluate them at each point until you find the
solution to the problem (which makes $\order{n^2}$ time complexity). However,
we can do a much better job.

\textbf{*Include Divided Difference formula on cheat sheet.} 

\soln If we do Lagrange interpolation of degree 4 on a degree 4 polynomial, we
will get $P = f$. \ie the best fourth degree interpolating polynomial for a
fourth degree polynomial is itself. Thus, the coefficient of $x^4 ( = 5)$ for
$P$ will be the same as the one for $f$.

The form of the Newton divided difference form of $f$ (section 3.4 pls review me)

\begin{align*}
  a_0 &+ a_1\pqty{x-x_0} + a_2\pqty{x-x_0}\pqty{x-x_1} \\
      &+ a_3\pqty{x-x_0}\pqty{x-x_1}\pqty{x-x_2}\pqty{x-x_3} \\
      &+ a_4\pqty{x-x_0}\pqty{x-x_1}\pqty{x-x_2}\pqty{x-x_3}\pqty{x-x_4}
\end{align*}

In this formula, $a_i$ are the coefficients to the interpolating polynomial
found using divided differences. Thus, $a_4 = f\bqty{x_0, x_1, x_2, x_3, x_4} =$
the coefficient of $x^4$ in $f$. Even if the polynomial is written differently,
the divided difference will still be the same (there are infinite ways to write
the same unique polynomial).

\section{Problem 4}

A natural cubic spline $S$ on $[1,3]$ is defined by
\[
  S = \begin{cases}
    S_0(x) = 1+a(x-1)+b(x-1)^3 & \text{if } 1\leq x< 2\\
    S_1(x) = 1+2(x-2)+c(x-2)^2 + d(x-2)^3 & \text{if } 2\leq x\leq 3
  \end{cases}
.\]
Find $a,b,c,d$.
\begin{align*}
  &S_0(2) = S_1(2) \\
  &1+a+b = 1 \implies a+b=0 \\
  &S_0'(2) = S_1'(2) \\
  &a+3b = 2 \implies \begin{cases}
  b = 1 \\
  a = -1 
  .\end{cases} \\
  &S_0''(2) = S_1''(2) \\
  &6b  = 2c \implies c = 3 \\
  &S''(3) = 0 \quad \text{endpoint condition S' = 0} \\
  &\implies d = -1
\end{align*}
Basically, every problem on the exam is going to be about the
\textbf{properties} of the different methods we learned in class.

*What on earth are splines? Check Lecture 23 notes.*

\section{Problem 5}

Derive an $\order{h^2}$ three-point formula to approximate $f'(x_0)$ that uses
$f(x_0), f(x_0+h), f(x_0+3h)$.

\soln Put a polynomial through your data points. Differentiate the polyomial.
That's your $\order{h^2}$ approximation. *Learn how to put a polynomial through
data points.*

\begin{align*}
  x_0 &= x_0 \\
  x_1 &= x_0 + h \\
  x_2 &= x_0 + 3h 
\end{align*}

\begin{align*}
  P_2(x) &= L_0(x) f_0 + L_1(x) f_1 + L_2(x) f_2 \\
         &= \frac{(x-x_1)(x-x_2)}{(x_0-x_1)(x_0-x_2)} f_0 \\
         &\quad + \frac{(x-x_0)(x-x_2)}{(x_1-x_0)(x_1-x_2)} f_1 \\
         &\quad + \frac{(x-x_0)(x-x_1)}{(x_2-x_0)(x_2-x_1)} f_2
\end{align*}

If it has zeroes at $x_1$ and $x_2$, then it must have the factors $x-x_1$ and
$x-x_2$.

\begin{align*}
  \left.P_2'(x)\right|_{x=x_0} &= \frac{(-h)+(-3h)}{(-h)(-3h)}f_0 
    + \frac{-3h}{h(-2h)}f_1 
    + \frac{-h}{(3h)(2h)}f_2
\end{align*}

\section{Problem 6}

Find the constants $c_0, c_1$ and $x_1$ so that the quadature formula 
\[
  \int_{-1}^{2} f(x) \, dx = c_0 f(0) + c_1 f(x_1)
.\]
has the highest possible degree of precision.

\[
f=1: \int_{-1}^{2} 1 \, dx = c_0 1 + c_1 1
.\]

\[
  f=x: \int_{-1}^{2} x \, dx = \left.\frac{1}{2}x^2\right|_{-1}^2 = c_1 x_1
.\]

\[
f=x^2: \int_{-1}^{2} x^2 \, dx = \left.\frac{1}{3}x^3\right|_{-1}^2 =
.\]

I didn't catch the rest of this problem.

\section{Problem 7}
Suppose $N(h)$ is an approximation to $M$ for every $h>0$ and that
\[
  M=N(h) + K_1h + K_2h^2 + \dots
.\]
for some constants $K_1, K_2, \dots$. Use the values $N(h)$ and $N(\frac{h}{4})$
to produce an $\order{h^2}$ approximation to $M$.

\begin{equation}
  M = N(h) + K_1h + K_2h^2 + \dots
  \label{eq:M_h}
.\end{equation}
\begin{equation}
  M = N(\frac{h}{4}) + K_1\frac{h}{4} + K_2\frac{h^2}{4} + \dots
  \label{eq:M_h_over_4}
.\end{equation}

We solve for 4 $\times$ \eqref{eq:M_h_over_4} $-$ \eqref{eq:M_h} to get

\[
  M = \boxed{\frac{4N(\frac{h}{4}) - N(h)}{3}} + \order{h^2}
.\]

Good luck to everyone on the final exam.

\end{document}
