\documentclass[12pt]{article}
\usepackage{amsmath}
\usepackage{amssymb}
\usepackage{amsthm}
\usepackage{amsfonts}
\usepackage{graphicx}
\usepackage{textcomp}
\usepackage{hyperref}
\usepackage{tikz}
\usepackage{enumitem}
\usepackage{mathtools}
\usepackage{enumitem}
\usepackage{wasysym}
\usepackage{ulem}
\usepackage{xspace}
\usepackage{csquotes}
\usepackage{booktabs}

\DeclareMathOperator{\dist}{dist}
\DeclareMathOperator{\Nul}{Nul}
\DeclareMathOperator{\Row}{Row}
\DeclareMathOperator{\proj}{proj}

\setlength{\arraycolsep}{12pt}

\newcommand{\defn}{\textbf{Def.}\xspace}
\newcommand{\thm}{\textbf{Thm.}\xspace}
\newcommand{\ex}{\textbf{ex.}\xspace}
\newcommand{\Ex}{\textbf{Ex.}\xspace}
\newcommand{\ie}{\textbf{i.e.}\xspace}
\newcommand{\lemma}{\textit{Lemma}\xspace}
\newcommand{\bproof}{\textit{Proof ($\impliedby$).}\xspace}
\newcommand{\fproof}{\textit{Proof ($\implies$).}\xspace}
\newcommand{\bigEps}{\mathcal{E}}
\newcommand{\soln}{\textit{Soln.}\xspace}

\renewcommand{\arraystretch}{1.25} % Adjust row spacing


\hypersetup{
    colorlinks=true,
    linkcolor=blue,
    filecolor=blue,      
    urlcolor=blue,
}

\newcommand{\ulhref}[2]{\href{#1}{\color{blue}\uline{#2}}}

\begin{document}

\title{MACM 316 Theorems}
\author{Alexander Ng}
\date{Wednesday, March 5, 2025}

\maketitle

\thm Let $A$ be an $n \times m$ matrix, $B$ be an $m \times k$ matrix, $C$ be an
$k \times p$ matrix, $D$ be an $m \times k$ matrix and $\lambda$ a real number.

\begin{enumerate}[label=(\alph*)]
  \item $A(BC) = (AB)C$ (associativity)
  \item $A(B+D) = AB + AD$ (distributivity)
  \item $IB = B$ and $BI = B$ (idempotency)
  \item $\lambda (AB) = (\lambda A)B = A(\lambda B)$ (scalar associativity)
\end{enumerate}

\defn An $n \times n$ matrix $A$ is said to be nonsingular (or invertible) if an
$n \times n$ matrix $A^{-1}$ exists with 

\[
  A A^{-1} = A^{-1} A = I
.\]

The matrix $A^{-1}$ is called the inverse of $A$. A singular matrix does not
have an inverse.

\thm For any nonsingular $n \times n$ matrix $A$

\begin{enumerate}[label=(\alph*)]
\item $A^{-1}$ is unique
\item $A^{-1}$ is nonsingular and $(A^{-1})^{-1} = A$
\item If $B$ is also a nonsingular $n \times n$ matrix, then $(AB)^{-1} =
  B^{-1}A^{-1}$
\end{enumerate}

You can find $A^{-1}$ by row reducing the augmented matrix $[A | I]$, which
looks like this:

\[
  \left[
    \begin{array}{ccc|ccc}
      a_{11} & a_{12} & a_{13} & 1 & 0 & 0 \\
      a_{21} & a_{22} & a_{23} & 0 & 1 & 0 \\
      a_{31} & a_{32} & a_{33} & 0 & 0 & 1
    \end{array} 
  \right]
\]


\end{document}
