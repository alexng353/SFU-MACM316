\documentclass[12pt]{article}
\usepackage{amsmath}
\usepackage{amssymb}
\usepackage{amsthm}
\usepackage{amsfonts}
\usepackage{graphicx}
\usepackage{textcomp}
\usepackage{hyperref}
\usepackage{tikz}
\usepackage{enumitem}
\usepackage{mathtools}
\usepackage{enumitem}
\usepackage{wasysym}
\usepackage{ulem}
\usepackage{xspace}
\usepackage{csquotes}

\DeclareMathOperator{\dist}{dist}
\DeclareMathOperator{\Nul}{Nul}
\DeclareMathOperator{\Row}{Row}
\DeclareMathOperator{\proj}{proj}

\setlength{\arraycolsep}{12pt}

\newcommand{\defn}{\textbf{Def.}\xspace}
\newcommand{\thm}{\textbf{Thm.}\xspace}
\newcommand{\ex}{\textbf{ex.}\xspace}
\newcommand{\Ex}{\textbf{Ex.}\xspace}
\newcommand{\ie}{\textbf{i.e.}\xspace}
\newcommand{\lemma}{\textit{Lemma}\xspace}
\newcommand{\bproof}{\textit{Proof ($\impliedby$).}\xspace}
\newcommand{\fproof}{\textit{Proof ($\implies$).}\xspace}

\renewcommand{\arraystretch}{1.25} % Adjust row spacing

\begin{document}

\title{MACM 316 Lecture 4 - Numerical Approximation and Big-O Notation}
\author{Alexander Ng}
\date{Monday, January 13, 2025}

\maketitle
\section{Linear Approximation of $\sqrt{16.1}$}

We approximate $\sqrt{16.1}$ without using the square root algorithm.

\subsection{Solution}
Let $f(x) = \sqrt{x}$ and choose an expansion point $x_0 = 16$, since 
$\sqrt{16} = 4 \in \mathbb{Z}$ is easily computable. Using the first-order 
Taylor approximation:
\begin{equation*}
    f(x_0 + h) \approx f(x_0) + h f'(x_0),
\end{equation*}
where $h = 0.1$.

\subsection{Computation}
\begin{align*}
    f(16) &= \sqrt{16} = 4, \\
    f'(x) &= \frac{1}{2\sqrt{x}}, \quad f'(16) = \frac{1}{8}, \\
    f(16.1) &\approx 4 + 0.1 \times \frac{1}{8} = 4.0125.
\end{align*}
The exact value is $4.01248052955\dots$, with a small truncation error.
Since the machine error is on the order of $10^{-14}$, it is negligible compared 
to the Taylor approximation error.

\section{Algorithm Quantification}

Numerical methods construct a sequence of better approximations, converging to 
a solution $\alpha$.

Given a sequence $\{\alpha_n\}$:
\begin{equation*}
    \lim_{n\to\infty} \alpha_n = \alpha.
\end{equation*}
We quantify convergence speed by analyzing $|\alpha - \alpha_n| \leq c$, where 
$c$ is a target error.

\subsection{Example: Convergence of $\sin(1/n)$}

Consider $\alpha_n = \sin(1/n)$, which converges to $\alpha = 0$ as $n \to \infty$.
We rewrite:
\begin{equation*}
    \lim_{n \to \infty} \sin(1/n) = \lim_{h \to 0} \sin(h),
\end{equation*}
which is easier to analyze. Expanding $\sin(h)$ in a Taylor series:
\begin{equation*}
    \sin(h) = h - \frac{h^3}{3!} + \frac{h^5}{5!} + \dots.
\end{equation*}
For small $h$, $\sin(h) \approx h$, implying:
\begin{equation*}
    |\alpha_n - \alpha| \leq \frac{1}{n}.
\end{equation*}
Thus, $\alpha_n$ converges to $\alpha = 0$ with rate of convergence $O(1/n)$.

\section{Big-O Notation}

For a sequence $\{A_n\}$, if:
\begin{equation*}
    |A_n - A| \leq k |B_n| \quad \text{for sufficiently large } n,
\end{equation*}
where $k$ is a constant, then we say:
\begin{equation*}
    A_n = A + O(B_n).
\end{equation*}

\subsection{Example: $\sin(1/n)$ Convergence}

From before, $|\alpha_n - \alpha| \leq 1/n$, so:
\begin{equation*}
    A_n = \sin(1/n) \text{ converges to } A = 0 
    \text{ with rate of convergence } O(1/n).
\end{equation*}

\subsection{Example: Convergence of $n \sin(1/n)$}

We evaluate:
\begin{equation*}
    \lim_{n \to \infty} n \sin(1/n) = 1.
\end{equation*}
Changing variables, $h = 1/n$, we obtain:
\begin{equation*}
    \lim_{h \to 0} \frac{\sin(h)}{h} = 1.
\end{equation*}
Expanding $\sin(h)/h$ in Taylor form:
\begin{equation*}
    \frac{\sin(h)}{h} = 1 - \frac{h^2}{6} + O(h^4).
\end{equation*}
For small $h$:
\begin{equation*}
    \frac{\sin(h)}{h} - 1 \approx -\frac{h^2}{6},
\end{equation*}
so $\alpha_n$ converges to $\alpha = 1$ with rate $O(1/n^2)$.

\section{Takeaways}

Big-O notation quantifies algorithm efficiency by ignoring constants and focusing 
on convergence trends. Constants vary across systems, so we care about general 
convergence patterns rather than specific values.
\end{document}
