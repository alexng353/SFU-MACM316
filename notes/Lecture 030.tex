\documentclass[12pt]{article}
\usepackage{amsmath}
\usepackage{amssymb}
\usepackage{amsthm}
\usepackage{amsfonts}
\usepackage{graphicx}
\usepackage{textcomp}
\usepackage{hyperref}
\usepackage{tikz}
\usepackage{enumitem}
\usepackage{mathtools}
\usepackage{enumitem}
\usepackage{wasysym}
\usepackage{ulem}
\usepackage{xspace}
\usepackage{booktabs}

\ifPDFTeX % ensure generation of machine readable output
\input{glyphtounicode}
\pdfgentounicode=1
\usepackage[T1]{fontenc}
\usepackage[utf8]{inputenc}
\usepackage{lmodern}
\fi

\usepackage{csquotes}

\DeclareMathOperator{\dist}{dist}
\DeclareMathOperator{\Nul}{Nul}
\DeclareMathOperator{\Row}{Row}
\DeclareMathOperator{\proj}{proj}

\setlength{\arraycolsep}{12pt}

\newcommand{\defn}{\textbf{Def.}\xspace}
\newcommand{\thm}{\textbf{Thm.}\xspace}
\newcommand{\ex}{\textbf{ex.}\xspace}
\newcommand{\Ex}{\textbf{Ex.}\xspace}
\newcommand{\ie}{\textbf{i.e.}\xspace}
\newcommand{\lemma}{\textit{Lemma}\xspace}
\newcommand{\bproof}{\textit{Proof ($\impliedby$).}\xspace}
\newcommand{\fproof}{\textit{Proof ($\implies$).}\xspace}
\newcommand{\bigEps}{\mathcal{E}}
\newcommand{\soln}{\textit{Soln.}\xspace}

\renewcommand{\arraystretch}{1.25} % Adjust row spacing

\hypersetup{
    colorlinks=true,
    linkcolor=blue,
    filecolor=blue,      
    urlcolor=blue,
}

\newcommand{\ulhref}[2]{\href{#1}{\color{blue}\uline{#2}}}

\begin{document}

\title{MACM 316 Lecture 30}
\author{Alexander Ng}
\date{Wednesday, March 26, 2025}

\maketitle

\section{Adaptive Quadrature}
\label{sec:adaptive_quadrature}

Composative quadrature rules necessitate the use of equally spaced points. This
does not take into account that some portions of the curve may have large
functional variations that require more attention than other portions of the
curve. It is useful to introduce a method that adjusts the step size to be
smaller over portions of the curve where a larger functional variation occurs.
Thi technique is called adaptive quadrature. We will now discuss an adaptive
based on Simpson's Rule. The other composite procedures can be modified in a
similar manner.

We want to approximate 

\begin{equation*}
  \int_{a}^{b} f(x) \, dx
\end{equation*}

to within a specified tolerance $\bigEps > 0$.

We will start by applying Simpson's Rule with a step size $h=\frac{(b-a)}{2}$

\begin{align*}
  \int_{a}^{b} f(x) \, dx &= S(a,b) - \frac{h^5}{90}f^{(4)} (\mu)\\
  \text{where} \quad  S(a,b) &= \frac{h}{3}\left[f(a) + 4f(a+h) +f(b)\right]
  \qquad (*)\\
    \text{and} \quad  \mu &\in (a,b)
.\end{align*}

We want to know if we should further subdivide the interval, so we need an
estimate for the error. Unfortunately, we don't know $f^{(4)}(\mu)$. Instead, we
will estimate the error using Simpson's Rule with a step size $\frac{(b-a)}{4}$

\begin{align*}
  \int_{a}^{b} f(x) \, dx &= \frac{h}{6} \left[
    f(a) + 4f(a+\frac{h}{2}) + 2f(a+h) +4f(a+ \frac{3h}{2}) + f(b)
  \right] \\
                          &- (\frac{h}{2})^4 \frac{(b-a)}{180} f^{(4)} (\tilde{\mu})
.\end{align*}

for some $\tilde{\mu} \in (a,b)$.

\uline{OR}

\begin{align*}
  \int_{a}^{b} f(x) \, dx &= S(a, \frac{(a+b)}{2}) + S(\frac{(a+b)}{2}, b) -
  \frac{1}{16} \frac{h^5}{90} f^{(4)} (\tilde\mu) \\
  \text{where} \quad  S(a, \frac{(a+b)}{2}) &= \frac{h}{6} \left[
    f(a) + 4f(a+\frac{h}{2}) + f(a+h)
  \right] \qquad (**)\\
    \text{and} \quad  S(\frac{(a+b)}{2},b) &= \frac{h}{6} \left[
      f(a+h) + 4f(a+\frac{3h}{2}) + f(b)
      \right]
\end{align*}

Now, we assume $f^{(4)}(\mu) \approx f^{(4)}(\tilde\mu)$.

\noindent
\uline{Note:} if $f^{(4)}$ is continuous, this assumption will hold for
sufficiently small $h$.

Subtract (**) from (*) to obtain

\[
0 \equiv S(a,b) - S\left(a, \frac{a+b}{2}\right) - S\left(\frac{a+b}{2}, b\right) - \frac{15}{16} \cdot \frac{h^5}{90} f^{(4)}(\mu)
\]
\[
\text{or} \frac{h^5}{90} f^{(4)}(\mu) \approx \frac{16}{15} \left[ \mathcal{I}(a,b) - \mathcal{I}\left(a, \frac{a+b}{2}\right) - \mathcal{I}\left(\frac{a+b}{2}, b\right) \right]
.\]

\begin{align*}
\therefore \quad \left| \text{error} \right| 
&= \left| \int_a^b f(x) \, dx - \mathcal{I}\left(a, \frac{a+b}{2}\right) - \mathcal{I}\left(\frac{a+b}{2}, b\right) \right| \\
&\equiv \frac{1}{16} \left| \frac{h^5}{90} f^{(4)}(\mu) \right| \\
&\equiv \frac{1}{15} \left| \mathcal{I}(a,b) - \mathcal{I}\left(a, \frac{a+b}{2}\right) - \mathcal{I}\left(\frac{a+b}{2}, b\right) \right|
\end{align*}

Thus if we want $|\text{error}|<\bigEps $, we typically insist that 

\[
\frac{1}{10} |S(a,b)-S(a,\frac{a+b}{2}) - S(\frac{a+b}{2}, b)| \leq \bigEps
.\]

Often, a factor of $\frac{1}{10}$ is used rather than $\frac{1}{15}$ since we
had to make the assumption that $f^{(4)}(\mu) \approx f^{(4)}(\tilde\mu)$ and we
prefer to be somewhat conservative in our error estimates.

From last day, we had an error estimate for Simpson's Rule: 

\[
E = \frac{1}{15} |S(a,b)-S(a,\frac{a+b}{2}) - S(\frac{a+b}{2}, b)|
.\]

\Ex Compute the Simpson's Rule approximations for

\[
\int_{a}^{1.5} x^2 \ln(x) \, dx = 0.19225935
.\]

and calculate the error estimate and the actual error.

In deriving our eror estimate, we had to make the assumption that $f^{(4)}(\mu)
\approx f^{(4)}(\tilde\mu)$, and to compensate for this error we typically
insist that

\[
\frac{1}{15} |S(a,b)-S(a,\frac{a+b}{2}) - S(\frac{a+b}{2}, b)| < \frac{2}{3}
\bigEps \equiv \tilde\bigEps
.\]

Adaptive quadrature methods are built on top of this condition. If the error
estimate is less than $\frac{2}{3}$ the desired tolerance $\bigEps$, then 

\[
S(a, \frac{(a+b)}{2}) + S(\frac{(a+b)}{2}, b)
.\]

is assumed to be a sufficiently accurate approximation to $\int_{a}^{b} f(x) \, dx$ 

Otherwise, we apply Simpson's Rule to the subintervals $[a, \frac{a+b}{2}]$ and
$[\frac{a+b}{2}, b]$. We also use the error estimation in each of the
subintervals. If each error estimate is less than $\frac{\bigEps}{2}$, then we
sum the approximations to get an approximation of $\int_{a}^{b} f(x) \, dx$.

If the approximation on one of the subintervals fails to be within the tolerance
$\frac{\tilde \bigEps}{2}$ then that subinterval is itself subdivided and the
procedure is reapplied to the two subintervals to determine if the approximation
on each subinterval is accurate to within $\frac{\tilde \bigEps}{2}$.

This recursive procedure is continued until each portion is within the desired
tolerance.

\section{Gaussian Quadrature}

Thus far we hvae onlyd ealt with quadrature formulae

\[
\int_{a}^{b} f(x) \, dx \approx \sum_{j=1}^{n} a_j f(x_j)
.\]

that relied on nodes that are equally spaced. This is a nice feature for
composite rules because it reduces the number of function evaluations. However,
if we allow ourselves to use unequally spaced points, then we can construct
quadrature formulas of higher order accuracy.

Notice that with $n$ nodes and $n$ weights we have $2n$ free parameters and we
may hope to find an optimal quadrature formula which is exact for polynomials of
degree $\leq 2n-1$.

First, note that an integral

\[
\int_{a}^{b} f(x) \, dx
.\]

over an interval $[a,b]$ can be transformed into an integral over $[-1, 1]$ by
using a change of variables

\begin{align*}
\text{let}\quad t &= \frac{2x-a-b}{b-a} \\
\text{then}\quad  x &= \frac{1}{2}[(b-a) + a + b] \\
\text{and} \quad \,dx &= \frac{1}{2}(b-a) \,dt
\end{align*}

So 

\[
  \int_{a}^{b} f(x) \, dx = \int_{-1}^{1} f(\frac{1}{2}[(b-a)t + a+b]) \,
  \frac{1}{2}(b-a)dt
.\]

so without loss of generality, we will consider integrals over the interval $[-1,1]$.

Suppose that $n=2$ (2 nodes) and that we want to determine $c_1, c_2, x_1, x_2$
so that the integration formula

\[
\int_{-1}^{1} f(x) \, dx = c_1 f(x_1) + c_2 f(x_2)
.\]

gives the exact result whenever $f(x)$ is a polynomial of degree $2(2)-1 =3$.

\[
\ie f(x) = a_0 + a_1 x + a_2 x^2 + a_3 x^3
.\]

Since 

\begin{align*}
&\int_{-1}^{1} a_0 + a_1 x + a_2 x^2 + a_3x^3 \, dx  \\
&= a_0 \int_{-1}^{1}  \, da + a_1 \int_{-1}^{1} x \, dx + a_2 \int_{-1}^{1} x^2 \, dx + a_3 \int_{-1}^{1} x^3 \, dx \\
\end{align*}

This problem is equivalent to showing that the formula is exact for $f(x) = 1,
x, x^2, x^3$.

cases:

\begin{center}
  \begin{tabular}{c|c}
    Function $f(x)$ & Equation \\
    \hline
    $1$ & $c_1 + c_2 = a_0 \int_{-1}^{1} \, dx$ \\
    $x$ & $c_1x_1 + c_2x_2 = \int_{-1}^{1} x \, dx$ \\
    $x^2$ & $c_1x_1^2 + c_2x_2^2 = \int_{-1}^{1} x^2 \, dx$ \\
    $x^3$ & $c_1x_1^3 + c_2x_2^3 = \int_{-1}^{1} x^3 \, dx$ \\
  \end{tabular}
\end{center}

Solving this system gives us that

\[
c_1 = 1, c_2 = 1, x_1 = - \frac{\sqrt{3}}{3}, x_2 = \frac{\sqrt{3}}{3}
.\]

$\implies$ the approximation formula is 
\[
  \int_{-1}^{1} f(x) \, dx = f\left(-\frac{\sqrt{3}}{3}\right) +
  f\left(\frac{\sqrt{3}}{3}\right)
.\]

This approach can be used to obtain the nodes and coefficients for larger $n$,
but Legendre polynomials can be used to obtain them more easily.

\subsection{Legendre Polynomials}

The Legendre Polynomials are defined according to the following two properties:

\begin{enumerate}
\item $P_n(x)$ is a polynomial of degree $n$
\item $\int_{-1}^{1} P(x)P_n(x) \, dx = 0$ whenever $P(x)$ is a polynomial of
  degree less than $n$.
\end{enumerate}

\end{document}
