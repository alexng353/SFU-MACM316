\documentclass[12pt]{article}
\usepackage{amsmath}
\usepackage{amssymb}
\usepackage{amsthm}
\usepackage{amsfonts}
\usepackage{graphicx}
\usepackage{textcomp}
\usepackage{hyperref}
\usepackage{tikz}
\usepackage{enumitem}
\usepackage{mathtools}
\usepackage{enumitem}
\usepackage{wasysym}
\usepackage{ulem}
\usepackage{xspace}
\usepackage{csquotes}

\DeclareMathOperator{\dist}{dist}
\DeclareMathOperator{\Nul}{Nul}
\DeclareMathOperator{\Row}{Row}
\DeclareMathOperator{\proj}{proj}

\setlength{\arraycolsep}{12pt}

\newcommand{\defn}{\textbf{Def.}\xspace}
\newcommand{\thm}{\textbf{Thm.}\xspace}
\newcommand{\ex}{\textbf{ex.}\xspace}
\newcommand{\Ex}{\textbf{Ex.}\xspace}
\newcommand{\ie}{\textbf{i.e.}\xspace}
\newcommand{\lemma}{\textit{Lemma}\xspace}
\newcommand{\bproof}{\textit{Proof ($\impliedby$).}\xspace}
\newcommand{\fproof}{\textit{Proof ($\implies$).}\xspace}

\renewcommand{\arraystretch}{1.25} % Adjust row spacing

\begin{document}

\title{MACM 316 Lecture 3 - Reducing Roundoff Error and Taylor Series Approximation}
\author{Alexander Ng}
\date{Friday, January 10, 2025}

\maketitle
\section{Reducing Roundoff Error}

One way to reduce roundoff error is to minimize the number of floating-point 
operations.

\subsection{Polynomial Evaluation Using Nested Multiplication}

Consider evaluating the polynomial:
\begin{equation*}
    f(z) = 1.01z^4 - 4.62z^3 - 3.11z^2 + 12.2z - 1.99.
\end{equation*}
We can rewrite this expression using nested multiplication:
\begin{align*}
    f(z) &= (1.01z^3 - 4.62z^2 - 3.11z + 12.2)z - 1.99 \\
         &= ((1.01z^2 - 4.62z - 3.11)z + 12.2)z - 1.99 \\
         &= \dots
\end{align*}
By factoring out $z$ as much as possible, we reduce the total number of 
floating-point operations, minimizing error accumulation.

\section{Cancellation Errors}

\subsection{Quadratic Formula and Cancellation Errors}

Consider solving the quadratic equation:
\begin{equation*}
    ax^2 + bx + c = 0.
\end{equation*}
Using the quadratic formula:
\begin{align*}
    x_1 &= \frac{-b + \sqrt{b^2 - 4ac}}{2a}, \\
    x_2 &= \frac{-b - \sqrt{b^2 - 4ac}}{2a}.
\end{align*}

Suppose $b = 600$, $a = c = 1$. The issue arises because $-b$ is close in 
magnitude to $+\sqrt{b^2 - 4ac}$, causing significant cancellation error in 
$x_1$.

\subsection{Reformulating to Reduce Cancellation}

We rationalize the numerator:
\begin{equation*}
    x_1 = \frac{(-b + \sqrt{b^2 - 4ac})}{2a} \times 
          \frac{(-b - \sqrt{b^2 - 4ac})}{(-b - \sqrt{b^2 - 4ac})}.
\end{equation*}

This simplifies to:
\begin{equation*}
    x_1 = \frac{b^2 - (b^2 - 4ac)}{2a(-b - \sqrt{b^2 - 4ac})}.
\end{equation*}
Now, the cancellation error is eliminated. If $b = -600$, the same issue 
occurs with $x_2$, and we apply the same rationalization technique.

\section{Review of Taylor Series}

Taylor’s theorem is fundamental for numerical approximations.

\subsection{Definition of Taylor Series}

Given a function $f(x)$ that is sufficiently smooth on $[a, b]$, we can 
approximate $f(x)$ with a Taylor polynomial $P_n(x)$:
\begin{equation*}
    P_n(x) = f(x_0) + f'(x_0)(x - x_0) + \frac{f''(x_0)}{2!} (x - x_0)^2 + 
             \dots + \frac{f^{(n)}(x_0)}{n!} (x - x_0)^n.
\end{equation*}

Here, $x_0, x \in [a, b]$.

\subsection{Conditions for Taylor Series Expansion}

For $f(x)$ to have a valid Taylor series expansion:
\begin{itemize}
    \item $f \in C^n[a, b]$ (i.e., $f$, $f'$, $f''$, ..., $f^n$ must be continuous).
    \item $f^{(n+1)}$ must exist on $[a, b]$.
\end{itemize}

\subsection{Error in Taylor Approximation}

The error in Taylor series approximation is given by:
\begin{equation*}
    f(x) = P_n(x) + R_n(x),
\end{equation*}
where the remainder term $R_n(x)$ satisfies:
\begin{equation*}
    R_n(x) = \frac{f^{(n+1)}(c)}{(n+1)!} (x - x_0)^{n+1},
\end{equation*}
for some $c \in (x_0, x)$. The approximation is most accurate when $x$ is 
close to $x_0$.

\section{Example: Third-Order Taylor Polynomial for $\sin(x)$}

Find $P_3(x)$ for $f(x) = \sin(x)$ centered at $x_0 = 0$.

\begin{align*}
    P_3(x) &= f(0) + f'(0)(x - 0) + \frac{f''(0)}{2!} (x - 0)^2 + 
             \frac{f'''(0)}{3!} (x - 0)^3 \\
           &= x - \frac{x^3}{6}.
\end{align*}

\subsection{Error Analysis}

The remainder term for $n = 3$ is:
\begin{equation*}
    R_3(x) = \frac{f^{(4)}(c)}{4!} x^4.
\end{equation*}
Since $f^{(4)}(x) = \sin(x)$, we have:
\begin{equation*}
    R_3(x) = \frac{\sin(c)}{24} x^4.
\end{equation*}
For $x = 0.1$:
\begin{equation*}
    |R_3(0.1)| \leq \frac{|\sin(0.1)|}{24} (0.1)^4 < 4.2 \times 10^{-7}.
\end{equation*}
This shows the high accuracy of the Taylor series approximation.
\end{document}
