\documentclass[12pt]{article}
\usepackage{amsmath}
\usepackage{amssymb}
\usepackage{amsthm}
\usepackage{amsfonts}
\usepackage{graphicx}
\usepackage{textcomp}
\usepackage{hyperref}
\usepackage{tikz}
\usepackage{enumitem}
\usepackage{mathtools}
\usepackage{enumitem}
\usepackage{wasysym}
\usepackage{ulem}
\usepackage{xspace}
\usepackage{csquotes}

\DeclareMathOperator{\dist}{dist}
\DeclareMathOperator{\Nul}{Nul}
\DeclareMathOperator{\Row}{Row}
\DeclareMathOperator{\proj}{proj}

\setlength{\arraycolsep}{12pt}

\newcommand{\defn}{\textbf{Def.}\xspace}
\newcommand{\thm}{\textbf{Thm.}\xspace}
\newcommand{\ex}{\textbf{ex.}\xspace}
\newcommand{\Ex}{\textbf{Ex.}\xspace}
\newcommand{\ie}{\textbf{i.e.}\xspace}
\newcommand{\lemma}{\textit{Lemma}\xspace}
\newcommand{\bproof}{\textit{Proof ($\impliedby$).}\xspace}
\newcommand{\fproof}{\textit{Proof ($\implies$).}\xspace}

\renewcommand{\arraystretch}{1.25} % Adjust row spacing


\hypersetup{
    colorlinks=true,
    linkcolor=blue,
    filecolor=blue,      
    urlcolor=blue,
}

\newcommand{\ulhref}[2]{\href{#1}{\color{blue}\uline{#2}}}

\begin{document}

\title{MACM 316 Lecture 15}
\author{Alexander Ng}
\date{Friday, February 7, 2025}

\maketitle

\subsection*{Bisection Method Example}

He starts with an example on the bisection method. I will provide one later on.

Lowkey I missed notes from page 5.7 to 5.10 but I'll add them later. (page 5 of
chapter 2 part 1 notes)

\section{Fixed Point Iteration}

We wish to find the roots of an equation $f(p) = 0$. We will be focusing on
methods that \uline{iterate} to find the root:

\[ p_{n+1} = g(p_n) \]

We start by considering the fixed point problem.

\defn A fixed point $p$ is the value of $p$ such that $g(p) = p$.

We can form a fixed pt. problem from a root finding problem.

\[ f(p) = 0 \]

Find a fixed point problem.

e.g. set $g(x) = f(x) + x$

$g(p) = f(p) + p$

$p = g(p)$

there are many choices of $g$

Try $g(x) = f^3(x) + x$

Notice that fixed point problems and root finding problems are equivalent. (???)

\subsection{Example}

We are given $g(p) = p$. Formulate a root finding problem.

\[ f(x) = g(x) - x \]

Now we have $f(p) = 0$. We now have a root finding problem.

\[ f(p) = 0 \implies g(p) = p \]

\ie $f$ has a root $p$ implies $g$ has a fixed point $p$.

\[
g(p) = p \implies f(p) = 0
.\]

\ie $g$ has a fixed point $p$ implies $f$ has a root $p$.


There are many possible choices for $g$: 
\uline{example} $g(x) -x = \left( f(x) \right)^3$

Our ultimate goal is to find functions with fixed points.

\section{Theorems}

\textbf{Existence and Uniqueness}

$\star$If $g\in C[a,b]$ and $g(x) \in [a, b]$ for all $x\in [a, b]$ then $g(x)$ has a
fixed point in $[a, b]$.

$\star\star$ Suppose, in addition, that $g'(x)$ exists on $(a, b)$ and that a
positive constant $k<1$ exists with 

\[
|g'(x)| \leq k < 1 \text{ for all } x \in (a, b)
.\]

then the fixed point in $[a, b]$ is unique

\proof($\star$): Existence

If $g(a) = a$ or $g(b) = b$ then $g$ has a fixed point at an 
endpoint.

Suppose not, then it must be true that $g(a) > a$ and $g(b) < b$.

Define $h(x) = g(x) - x$. Then $h$ is continuous on $[a, b]$ (adding two
continuous functions yields a continuous function) and

\[
h(a) = g(a) -a > 0 \text{ and } h(b) = g(b) - b < 0
.\]

The \textbf{IVT} implies that there exists a $p \in (a, b)$ for which $h(p) = 0$

This $g(p) - p = 0 \implies p$ is a fixed point of $g$

\proof($\star\star$): Uniqueness

Suppose, in addition,

\[
\forall x \in (a,b), |g'(x)| \leq k < 1
.\]

and that $p$ and $q$ are both fixed points in $[a, b]$ with $p \ne q$.

By the \textbf{MVT}, a number $c$ eixsts between $p$ and $q$ such that

\[
\frac{g(p) - g(q)}{p-q} = g'(c)
.\]

then 

\begin{align*}
  |p-q| &= |g(p) - g(q)| \\
  &= |g'(c)||p-q| \\
  &\leq k|p-q| \\
  &< |p-q|  \\
\end{align*}

Which is a contradiction

This contradiction must come from the assumption that $p \ne q$

$\therefore p = q$ and the fixed point is unique.

We want to approximate the fixed point of a function $g$.

\textbf{IDEA}

\begin{itemize}
\item choose an initial approximation $p_0$
\item generate a squence $\displaystyle \left\{ p_n \right\}_{n=0}^\infty$
  such that $p_n = g(p_{n-1}); n \geq 1$
\end{itemize}

\uline{If} the sequence converges to $p$ and $g$ is continuous;

\begin{align*}
  p &\equiv \lim_{n\to\infty} p_n \\
  &= \lim_{n\to\infty} g(p_{n-1}) \\
  &=  g(\lim_{n\to\infty} p_{n-1}) \\
  &= g(p) \\
\end{align*}

This gives the \uline{Fixed Point Algorithm}:

*See next notes package*

\end{document}
