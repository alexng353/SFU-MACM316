\documentclass[12pt]{article}
\usepackage{amsmath}
\usepackage{amssymb}
\usepackage{amsthm}
\usepackage{amsfonts}
\usepackage{graphicx}
\usepackage{textcomp}
\usepackage{hyperref}
\usepackage{tikz}
\usepackage{enumitem}
\usepackage{mathtools}
\usepackage{enumitem}
\usepackage{wasysym}
\usepackage{ulem}
\usepackage{xspace}
\usepackage{csquotes}

\DeclareMathOperator{\dist}{dist}
\DeclareMathOperator{\Nul}{Nul}
\DeclareMathOperator{\Row}{Row}
\DeclareMathOperator{\proj}{proj}
\DeclareMathOperator{\fl}{fl}

\setlength{\arraycolsep}{12pt}

\newcommand{\defn}{\textbf{Def.}\xspace}
\newcommand{\thm}{\textbf{Thm.}\xspace}
\newcommand{\ex}{\textbf{ex.}\xspace}
\newcommand{\Ex}{\textbf{Ex.}\xspace}
\newcommand{\ie}{\textbf{i.e.}\xspace}
\newcommand{\lemma}{\textit{Lemma}\xspace}
\newcommand{\bproof}{\textit{Proof ($\impliedby$).}\xspace}
\newcommand{\fproof}{\textit{Proof ($\implies$).}\xspace}

\renewcommand{\arraystretch}{1.25} % Adjust row spacing

\begin{document}

\title{MACM 316 Lecture 2 - More Computer Arithmetic}
\author{Alexander Ng}
\date{Wednesday, January 8, 2025}

\maketitle

\section{Floating Point Decimal Normalization}

Can we write all real numbers in normalized scientific notation?
\begin{align*}
    732.5051 &\rightarrow +0.7325051 \times 10^{+3} \\
    -0.005612 &\rightarrow -0.5612 \times 10^{-2}
\end{align*}
For $x \in \mathbb{R}$, we can express it as:
\begin{equation*}
    x = \pm r \times 10^{\pm n}, \quad \text{where } \frac{1}{10} \leq r \leq 1.
\end{equation*}

In binary, we write:
\begin{equation*}
    x = \pm q \times 2^{\pm m}, \quad \text{where } \frac{1}{2} \leq q < 1.
\end{equation*}
Here, $q$ is the mantissa and $m$ is the integer exponent.

We limit $r$ and $q$ so that when $k < 1/\text{BASE}$, we can shift the decimal 
place and normalize the number further. When we have $1.x$, we rewrite it 
as $0.x \times \text{base}^1$.

\section{Rounding or Chopping (Sources of Error)}

Given $x = 0.a_1 a_2 \dots a_n a_{n+1} \dots a_m$ using $m$ digits, rounding 
to $n$ places follows:
\begin{itemize}
    \item If $0 \leq a_{n+1} < 5$, then $x = 0.a_1 a_2 \dots a_n$.
    \item If $5 \leq a_{n+1} \leq 9$, then $x = 0.a_1 a_2 \dots (a_n + 1)$.
\end{itemize}

\textbf{Example:}
\begin{align*}
    \text{round}(0.125) &= 0.13, \\
    \text{round}(-0.125) &= -0.13.
\end{align*}

Instead of rounding, truncation follows:
\begin{equation*}
    x = 0.a_1 a_2 \dots a_n.
\end{equation*}

Truncation introduces larger errors but is computationally cheaper than 
rounding.

\section{Error}

We define:
\begin{itemize}
    \item Absolute error: $|p - p^*|$.
    \item Relative error: $\frac{|p - p^*|}{|p|}$.
\end{itemize}
Absolute error is used when magnitude matters, particularly for small values. 
Relative error is preferred when values differ in scale.

\textbf{Example:}
\begin{align*}
    \text{Exact: } & 0.1, \quad \text{Approximate: } 0.099, \\
    \text{Relative Error: } & \frac{|0.1 - 0.099|}{0.1} = 0.01.
\end{align*}

\begin{center}
\begin{tabular}{|c|c|c|}
    \hline
    $t$ & $5 \times 10^{-t}$ & \text{Is error within bound?} \\
    \hline
    0 & 5 & $\checkmark$ \\
    1 & 0.5 & $\checkmark$ \\
    2 & 0.05 & $\checkmark$ \\
    3 & 0.005 & $\times$ \\
    \hline
\end{tabular}
\end{center}

Since $0.01 < 5 \times 10^{-2}$ but not $5 \times 10^{-3}$, we have two 
significant digits.

\section{Computations and Machine Representation}

Let $\fl(x)$ denote the machine representation of $x$. Computations on a 
machine follow:
\begin{equation*}
    \fl(\fl(x) + \fl(y)).
\end{equation*}
Each step introduces an error.

\textbf{Example:}
\begin{align*}
    p &= 0.54617, \quad q = 0.54601, \\
    r &= p - q = 0.00016.
\end{align*}
With 4-digit rounding,
\begin{align*}
    p^* &= 0.5462, \quad q^* = 0.5460, \\
    r^* &= p^* - q^* = -0.0002.
\end{align*}
Relative error:
\begin{equation*}
    \frac{|r - r^*|}{|r|} = 0.25.
\end{equation*}
A high relative error results when subtracting close numbers.

\section{Minimizing Error}

Consider computing $f(x) = \frac{1 - \cos x}{x^2}$ for $\bar{x} = 1.2 \times 10^{-5}$.

With 10-digit rounding:
\begin{align*}
    c &= \fl(\cos \bar{x}) = 0.9999999999, \\
    1 - c &= 0.0000000001.
\end{align*}
This results in a large error.

Using $\cos x = 1 - 2\sin^2(x/2)$:
\begin{equation*}
    f(x) = \frac{1}{2} \left( \frac{\sin(x/2)}{x/2} \right)^2.
\end{equation*}
This provides a more accurate computation.

\textbf{Conclusion:} Avoid subtracting close numbers. Use alternative 
representations like Taylor series or trigonometric identities.

\end{document}
