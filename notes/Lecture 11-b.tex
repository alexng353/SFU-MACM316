\documentclass[12pt]{article}
\usepackage{amsmath}
\usepackage{amssymb}
\usepackage{amsthm}
\usepackage{amsfonts}
\usepackage{graphicx}
\usepackage{textcomp}
\usepackage{hyperref}
\usepackage{tikz}
\usepackage{enumitem}
\usepackage{mathtools}
\usepackage{enumitem}
\usepackage{wasysym}
\usepackage{ulem}

\DeclareMathOperator{\dist}{dist}
\DeclareMathOperator{\Nul}{Nul}
\DeclareMathOperator{\Row}{Row}
\DeclareMathOperator{\proj}{proj}

\begin{document}

\renewcommand{\arraystretch}{1.25} % Adjust row spacing
\setlength{\arraycolsep}{12pt}

\title{MACM 316 Lecture 11 - Chapter 7}
\author{Alexander Ng}
\date{Wednesday, January 29, 2025}

\maketitle

\section{Iterative Techniques in Matrix Algebra}

We are interested in solving large linear systems $Ax=b$.

Suppose the matrix $A$ has a high $(>99.9\%)$ sparsity, i.e., most of the
entries are zeros. We would like to take advantage of this spare structure to
reduce the amount of computational work required. Unfortunately, Gaussin
Elimination is often unable to take advantage of the sparse structure. For this
reason, we consider \uline{iterative techniques}.

\section{Vector Norms}

To estimate how well a particular iterative technique approximates
the true solution, we will need some measurement of distance. This motivates
the notion of the vector norm.

\textbf{Def.} A vector norm on $\mathbb{R}^n$ is a function $||\cdot||$ from
$\mathbb{R}^n \to \mathbb{R}^n $ satisfying the following properties:

\begin{itemize}
  \item $||x|| \geq 0$ for all $x \in \mathbb{R}^n$
  \item $||x|| = 0 \iff x = \mathbf{0}$
  \item $||\alpha x|| = |\alpha| ||x||$ for all $\alpha \in \mathbb{R}$ and
    $x \in \mathbb{R}^n$
  \item $||x+y|| \leq ||x|| + ||y|| $ for all $x,y \in \mathbb{R}^n$
\end{itemize}

\textbf{Def. 2.} The $l_2$ or Euclidean norm of the vector $x$ is given by
\begin{equation*}
  ||x||_2 = \left\{ \sum_{i = 1}^{n}  x_i^2 \right\}^{1/2}
\end{equation*}

This represents the usual notion of distance.

\textbf{Def. 3.} The infinity or max norm of a vector $x$ is given by

\begin{equation*}
  ||x||_\infty = \max_{i=1}^n |x_i|
\end{equation*}

\textbf{Def. 4.} If $x,y \in \mathbb{R}^2$, then the $l_2$ distance between
$x$ and $y$ is given by

\begin{equation*}
  ||x-y||_2 = \left\{ \sum_{i=1}^{n} (x_i-y_i)^2 \right\}^{1/2}
.\end{equation*}

and the $l_\infty$ distance between $x$ and $y$ is given by

\begin{equation*}
  ||x-y||_\infty = \max_{i=1}^n |x_i-y_i|
.\end{equation*}

\section{Sequences?}

Iterative techniques generate a sequence of vectors.

\textbf{Def.} A sequence $\left\{ x^k \right\}_{k=1}^\infty$ of vectors in
$\mathbb{R}^n$ is said to converge to $x$ with respect to the norm $||\cdot||$
if, given any $\epsilon > 0$, there exists an integer $N('\epsilon)$ such that

\begin{equation*}
  ||x^{(k)} - x|| < \epsilon \text{ for all } k \geq N
\end{equation*}

*The notation $N('\epsilon)$ is used to emphasize that $N$ is dependent on 
$\epsilon$, however, $N$ is not a function of $\epsilon$.*

\subsection{Thm.}

The sequence of vectors $\left\{ x^k \right\}$ converges to $x$ in $\mathbb{R}^n $
with respect to $||\cdot||_\infty$ if and only if 
$\lim_{k \to \infty} x_i^{(k)} = x_i$. for each $i$.

\subsection{Thm. 2}

For each $x\in \mathbb{R}^n $

\begin{equation*}
  ||x||_\infty \leq ||x||_2 \leq \sqrt{n} ||x||_\infty
\end{equation*}

This theorem relates the infinity norm to the Euclidean norm, which is very 
useful in the context of iterative techniques.

\end{document}
