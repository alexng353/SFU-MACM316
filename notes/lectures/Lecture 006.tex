\documentclass[12pt]{article}
\usepackage{amsmath}
\usepackage{amssymb}
\usepackage{amsthm}
\usepackage{amsfonts}
\usepackage{graphicx}
\usepackage{textcomp}
\usepackage{hyperref}
\usepackage{tikz}
\usepackage{enumitem}
\usepackage{mathtools}
\usepackage{enumitem}
\usepackage{wasysym}
\usepackage{ulem}

\DeclareMathOperator{\dist}{dist}
\DeclareMathOperator{\Nul}{Nul}
\DeclareMathOperator{\Row}{Row}
\DeclareMathOperator{\proj}{proj}

\begin{document}

\renewcommand{\arraystretch}{1.25} % Adjust row spacing
\setlength{\arraycolsep}{12pt}

\title{MACM 316 Lecture 6}
\author{Alexander Ng}
\date{January 17, 2025}

\maketitle

\section{Partial Pivoting}

Parital pivoting is the simplest technique to avoid generating massive roundoff
errors in the Gaussian Elimination algorithm.

The idea is to find the largest element beneath the pivot and swap its row with
the pivot row.

Parital pivoting is sufficient for most linear systems. However, it can be
inadequate for certain problems.

\subsection{Example}

Consider the linear system:

\quad $E_1: 30.00x_1 + 591400x_2 = 491700$

\quad $E_2: 5.291x_1 - 6.130x_2 = 46.78$

No row exchanges are carried out during partial pivoting.

Now, the multiplier is $m_{21} = \frac{5.291}{30.000} = 0.1764$ and 
($E_2-m_{21}E_1 \to E_1$) gives the system

\quad $30.00x_1 + 591400x_2 \approx 591700$

\quad $-104300x_1 \approx -104400$

\end{document}
