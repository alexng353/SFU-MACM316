\documentclass[12pt]{article}
\usepackage{amsmath}
\usepackage{amssymb}
\usepackage{amsthm}
\usepackage{amsfonts}
\usepackage{graphicx}
\usepackage{textcomp}
\usepackage{hyperref}
\usepackage{tikz}
\usepackage{enumitem}
\usepackage{mathtools}
\usepackage{enumitem}
\usepackage{wasysym}
\usepackage{ulem}
\usepackage{xspace}
\usepackage{csquotes}
\usepackage{booktabs}

\DeclareMathOperator{\dist}{dist}
\DeclareMathOperator{\Nul}{Nul}
\DeclareMathOperator{\Row}{Row}
\DeclareMathOperator{\proj}{proj}

\setlength{\arraycolsep}{12pt}

\newcommand{\defn}{\textbf{Def.}\xspace}
\newcommand{\thm}{\textbf{Thm.}\xspace}
\newcommand{\ex}{\textbf{ex.}\xspace}
\newcommand{\Ex}{\textbf{Ex.}\xspace}
\newcommand{\ie}{\textbf{i.e.}\xspace}
\newcommand{\lemma}{\textit{Lemma}\xspace}
\newcommand{\bproof}{\textit{Proof ($\impliedby$).}\xspace}
\newcommand{\fproof}{\textit{Proof ($\implies$).}\xspace}
\newcommand{\bigEps}{\mathcal{E}}
\newcommand{\soln}{\textit{Soln.}\xspace}

\renewcommand{\arraystretch}{1.25} % Adjust row spacing


\hypersetup{
    colorlinks=true,
    linkcolor=blue,
    filecolor=blue,      
    urlcolor=blue,
}

\newcommand{\ulhref}[2]{\href{#1}{\color{blue}\uline{#2}}}

\begin{document}

\title{MACM 316 Lecture 24 - Chapter 3}
\author{Alexander Ng}
\date{Wednesday, March 12, 2025}

\maketitle

Even if we enforce the constraint that $f$ and $f'$ agree at the nodes, we still
have the problem that $P_n$ will be expensive to compute for large $n$ and that
for high $n$, the interpolating polynomial can still oscillate wildly.

\section{Splines (15.1)}

Previously, we used a single polynomial to compute an approximation to a
function over some finite interval. We increased the degree of the polynomial if
we wanted more accuracy. However, increasing the degree of the polynomial causes
oscillatory behaviour, and high degree polynomials have the property that a
fluctuation over a small portion of the interval can induce large fluctuations
over the entire interval.

An alternative approach is to divide the interval into subintervals and use a
different, lower degree polynomial in each subinterval. These polynomials are
patched together to give a piecewise polynomial approximation. 

\subsection{Possible Choices}
 
\subsubsection{Piecewise Linear Interpolation}

We join a set of data points by a series of straight lines.

The biggest disadvantage of this approach is that the approximation is typically
not smooth. \ie it is not differentiable at these points, which may not be
satisfactory.

\subsubsection{Piecewise Polynomails of Hermite Type}

*Hermite is pronounced \enquote{her-meat}

If we know the value of a function \textbf{and} it's derivative at each of the
data points, then we could use a Hermite cubic polynomial on each interval
$[x_i, x_{i+1}]$ to approximate the function. Often, however, the derivative of
the function is not known at the data points.

\subsubsection{Piecewise Quadratic Polynomials}

Alternatively, we could join a set of polynomials with quadratic polynomials.
This gives us $3$ arbitrary constants. There will be $2$ conditions to fit the
curve through the endpoints of each interval, but there isn't enough flexibility
to set conditions on the derivative at both endpoints $x$ and $x_n$.

We have enough flexibility to set conditions for the first $n-1$ data points,
but the data point $x_n$ requires four conditions, when we can only provide
three degrees of freedom.

\begin{itemize}
\item $P_0$: 2 endpoint conditions + 1 slope condition
\item $P_1$: 2 endpoint conditions + 1 slope conditions
\item $P_n$: 2 endpoint conditions + 2 slope conditions
\end{itemize}

\subsubsection{Cubic Spline Interpolation}

If we use cubic polynomials between each successive pair of nodes, then there
are four constants and it is possible to ensure that the interpolant

\begin{itemize}
  \item agrees with the function at all the nodes
  \item is continuously differentiable
  \item has continuous second derivatives
\end{itemize}

Given a function $f$ defined on $[a, b]$ and a set of nodes
$a=x_0<x_1<\dots<x_n=b$, a cubic spline interpolant $S$ for $f$ is a function
that satisfies the following:

\begin{itemize}
  \item $S(x) = S_j(x)$ on $[x_j, x_{j+1}]$
  \item $S_j(x_{j+1}) = f(x_{j+1}) = S_{j+1}(x_j)$ splines must pass through the
    data points
  \item $S_j'(x_{j+1}) = S'_{j+1}(x_{j+1})$ first derivative is continuous
  \item $S_j''(x_{j+1}) = S''_{j+1}(x_{j+1})$ second derivative is continuous
\end{itemize}

We also need a boundary condition. Typically one of the following hold in the
problems we will consider: 

\begin{itemize}
  \item $S'(x_0) = f'(x_0)$ and $S'(x_n) = f'(x_n)$ 
    \enquote{Clamped Boundary Conditions}
  \item $S''(x_0) = 0 = S''(x_n)$
    \enquote{Free or Natural Boundary Conditions} \\
    \small{
    *Natural Boundary Conditions tell us that $S$ is not curved at the endpoints.
  }
\end{itemize}

Of course, for the clamped case, we need derivative information, either from the
physics or from some assumption. If there are $(n+1)$ points, the number of
intervals and the number of $S_i(x)$'s are $n:$

\[
  S_i(x) = a_i + b_i (x-x_i) + c_i (x-x_i)^2 + d_i (x-x_i)^3
.\]

Letting $h_i = x_{i+1} - x_i$  we can simplify by substituting $x_{i+1}$ and
$x_i$ into $S_i(x)$, $S_i'(x)$ and $S_i''(x)$. Eliminating the $b_i$ and $d_i$
gives a linear system of equations:

\begin{align*}
&h_{i-1}c_{i-1} + 2(h_{i-1} + h_i)c_i + h_0 c_{i+1} \\
&\quad= \frac{3}{h_i}(a_{i+1}-a_i) - \frac{3}{h_{i-1}}(a_i - a_{i-1})\\
&\quad 1 \leq i \leq n-1
\end{align*}

where the $a_i$ are known

\[
  a_i = f(x_i) \qquad 0 \leq i \leq n-1
.\]

and we define 

\begin{align*}
  a_n &\equiv f(x_n) \\
  b_n &\equiv f'(x_n) \\
  c_n &\equiv \frac{f''(x_n)}{2}
\end{align*}

The $b_i$ and $d_i$ are easily found:

\[
  b_i = \frac{1}{h_i}(a_{i+1} - a_i) - \frac{h_i}{3} (2c_i + c_{i+1}) \quad 0 \leq i \leq n-1
.\]

\[
  d_i = \frac{c_{i+1} - c_i}{3h_i}
.\]

We still need to impose the boundary conditions:

\textbf{Natural Boundary Conditions}
Consider the Natural BC's: $S''(x_0) = S''(x_n) = 0$.

\[
\therefore C_n = 0 \quad\text{(by definition)}
.\]

\[
S_0''(x) = 2C_0 + 6d_0(x-x_0) \therefore S''(x_0) = 2c_0 + 6d_0(x_0-x_0) = 0 =
c_0
.\]

We can derive a matrix equation for the $[c_i]$:

\[
Ax = b
.\]

where 

\[
  A = \begin{bmatrix}
    1 & 0 & 0 & \dots & \dots &  0 \\
    h_0 & 2(h_0+h_1) & h_1 & \ddots & &  \vdots\\
    0 & h_1 & 2(h_1+h_2) & h_2 & \ddots  & \vdots\\
    \vdots & & \ddots & \ddots & \ddots  & \vdots\\
    \vdots & & & h_{n-2} & 2(h_{n-2}+h_{n-1}) & h_{n-1}  \\
    0 & \dots & \dots & 0 & 0 & 1
  \end{bmatrix}
\]

\[
 x=\begin{bmatrix}
 c_0\\
 c_1\\
 \vdots\\
 c_n\\
 \end{bmatrix}
.\]

\[
b=\begin{bmatrix}
0\\
\frac{3}{h_1}(a_2-a_1)-\frac{3}{h_0}(a_1-a_0)\\
\vdots\\
\frac{3}{h_{n-1}}(a_{n}-a_{n-l})-\frac{3}{h_{n-2}}(a_{n-1}-a_{n-2})\\
0
\end{bmatrix}
.\]

There exists a unique solution for the $c_i$'s.

The matrix is strictly diagonally dominant, so it is invertible and a unique
solution for the $c_i$ exists.

First we need a definition and a theorem.

\defn The $n \times n$ matrix $A$ is strictily diagonally dominant if 

\[
  |a_{ii}| > \sum_{j=1; j\neq i}^{n} |a_{ij}|
.\]

holds for each $i = 1, \dots, n$.

\thm A strictly diagonally dominant matrix is invertible.

Also note that the matrix $A$ is \textbf{tridiagonal}: all entires are zero
except for a band which is 3 entries wide centred on the main diagonal.

Solutions to tridiagonal linear systems can be found very efficiently:

*Only O(n) operations are needed using methods we shall discuss later

\textbf{Clamped Boundary Conditions}
We also want to treat clamped boundary conditions:

\[
f'(a) = S'(a) = S_0'(x_0) = b_0 + 2c_0 (x_0-x_0) + 3d_0 (x_0-x_0)^2 = b_0
.\]

but 
\[
  b_{i-1} = \frac{a_i - a_{i-1}}{h_{i-1}} - \frac{h_{i-1}}{3}(2c_{i-1} + c_i)
.\]

\[
\therefore b_0 = \frac{a_1 - a_0}{h_0} - \frac{h_0}{3}(2c_0 + c_1)
.\]

\[
\implies 2h_0c_0 + h_0 c_1 = \frac{3}{h_0}(a_1 - a_0) - 3f'(a)
.\]

Similarly, 

\[
  h_{n-1} c_{n-1} + 2h_{n-1} c_n = 3f'(b) - \frac{3}{h_{n-1}}(a_{n-1} - a_n)
.\]

Once again, we obtain a strictly diagonally dominant linear system

\begin{center}
  $\implies$ A unique solution exists for the $c_i$'s
\end{center}

Furthermore, the system is tridiagonal so it can be solved efficiently for the
coefficients of the spline.

\end{document}
