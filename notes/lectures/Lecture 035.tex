\documentclass[12pt]{article}
\usepackage{amsmath}
\usepackage{amssymb}
\usepackage{amsthm}
\usepackage{amsfonts}
\usepackage{graphicx}
\usepackage{textcomp}
\usepackage{hyperref}
\usepackage{tikz}
\usepackage{enumitem}
\usepackage{mathtools}
\usepackage{enumitem}
\usepackage{wasysym}
\usepackage{ulem}
\usepackage{xspace}
\usepackage{booktabs}
\usepackage{physics}

\ifPDFTeX % ensure generation of machine readable output
\input{glyphtounicode}
\pdfgentounicode=1
\usepackage[T1]{fontenc}
\usepackage[utf8]{inputenc}
\usepackage{lmodern}
\fi

\usepackage{csquotes}

\DeclareMathOperator{\dist}{dist}
\DeclareMathOperator{\Nul}{Nul}
\DeclareMathOperator{\Row}{Row}
\DeclareMathOperator{\proj}{proj}

\setlength{\arraycolsep}{12pt}

\newcommand{\Eg}{\textbf{Eg.}\xspace}
\newcommand{\Ex}{\textbf{Ex.}\xspace}
\newcommand{\Ie}{\textbf{I.e.}\xspace}
\newcommand{\bigEps}{\mathcal{E}}
\newcommand{\bproof}{\textit{Proof ($\impliedby$).}\xspace}
\newcommand{\defn}{\textbf{Def.}\xspace}
\newcommand{\eg}{\textbf{e.g.}\xspace}
\newcommand{\ex}{\textbf{ex.}\xspace}
\newcommand{\fproof}{\textit{Proof ($\implies$).}\xspace}
\newcommand{\ie}{\textbf{i.e.}\xspace}
\newcommand{\lemma}{\textit{Lemma}\xspace}
\newcommand{\soln}{\textit{Soln.}\xspace}
\newcommand{\thm}{\textbf{Thm.}\xspace}

\renewcommand{\arraystretch}{1.25} % Adjust row spacing

\hypersetup{
    colorlinks=true,
    linkcolor=blue,
    filecolor=blue,      
    urlcolor=blue,
}

\newcommand{\ulhref}[2]{\href{#1}{\color{blue}\uline{#2}}}

\begin{document}

\title{MACM 316 Lecture 35 - End of Chapter 5}
\author{Alexander Ng}
\date{Monday, April 7, 2025}

\maketitle

\section{Higher order Runge-Kutta Methods (24.8)}
Third order Runge-Kutta methods are not commonly used. However, fourth order
Runge-Kutta methods are widely used and derived in a similar fashion. Greater
complexity results from having to compare terms through $h^4$ and this gives a
set of 11 equations in 13 unknowns. The set of equations can be solved with 2
unknowns being chosen arbitrarily. 

The most commonly used set of values leads to the following algorithm:
\begin{align*}
  w_{n+1} &= w_n+\frac{1}{6}\bqty{k_1+2k_2+2k_3+k_4} \\
  k_1 &= hf(t_n, w_n) \\
  k_2 &= hf\bqty{t_n+\frac{1}{2}h, w_n + \frac{1}{2}k_1} \\
  k_3 &= hf\bqty{t_n+\frac{1}{2}h, w_n + \frac{1}{2}k_2} \\
  k_4 &= hf\bqty{t_n+h, w_n + k_3}
\end{align*}

The main computational effort in applying Runge-Kutta methods is the evaluation
of $f$. In the second order methods, the local truncation error is $\order{h^2}$
and the cost is two functional evaluations per step. The Runge-Kutta method of
order four requires four evaluations per step and the local truncation error is
$\order{h^4}$.

We may wonder about higher order formulas\dots

\subsection{Error Analysis of Higher Order Runge-Kutta Methods (24.9)}
Butcher has shown that the following relationship holds:

\begin{tabular}{c|ccccccc}
  evaluations & $2$ & $3$ & $4$ & $5\leq n\leq7$ & $8\leq n\leq9$ & $10\leq n$\\
  Best LTE & $O(h^2)$ & $O(h^3)$ & $O(h^4)$ & $O(h^{n-1})$ &
  $O(h^{n-2})$ &$O(h^{n-3})$
\end{tabular}

This indicates why methods of order $\leq 5$ are often used rather than higher
order methods with a larger step size.

\end{document}
