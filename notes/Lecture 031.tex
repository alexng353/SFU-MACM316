\documentclass[12pt]{article}
\usepackage{amsmath}
\usepackage{amssymb}
\usepackage{amsthm}
\usepackage{amsfonts}
\usepackage{graphicx}
\usepackage{textcomp}
\usepackage{hyperref}
\usepackage{tikz}
\usepackage{enumitem}
\usepackage{mathtools}
\usepackage{enumitem}
\usepackage{wasysym}
\usepackage{ulem}
\usepackage{xspace}
\usepackage{booktabs}

\ifPDFTeX % ensure generation of machine readable output
\input{glyphtounicode}
\pdfgentounicode=1
\usepackage[T1]{fontenc}
\usepackage[utf8]{inputenc}
\usepackage{lmodern}
\fi

\usepackage{csquotes}

\DeclareMathOperator{\dist}{dist}
\DeclareMathOperator{\Nul}{Nul}
\DeclareMathOperator{\Row}{Row}
\DeclareMathOperator{\proj}{proj}

\setlength{\arraycolsep}{12pt}

\newcommand{\defn}{\textbf{Def.}\xspace}
\newcommand{\thm}{\textbf{Thm.}\xspace}
\newcommand{\ex}{\textbf{ex.}\xspace}
\newcommand{\Ex}{\textbf{Ex.}\xspace}
\newcommand{\ie}{\textbf{i.e.}\xspace}
\newcommand{\lemma}{\textit{Lemma}\xspace}
\newcommand{\bproof}{\textit{Proof ($\impliedby$).}\xspace}
\newcommand{\fproof}{\textit{Proof ($\implies$).}\xspace}
\newcommand{\bigEps}{\mathcal{E}}
\newcommand{\soln}{\textit{Soln.}\xspace}

\renewcommand{\arraystretch}{1.25} % Adjust row spacing

\hypersetup{
    colorlinks=true,
    linkcolor=blue,
    filecolor=blue,      
    urlcolor=blue,
}

\newcommand{\ulhref}[2]{\href{#1}{\color{blue}\uline{#2}}}

\begin{document}

\title{MACM 316 Lecture 31}
\author{Alexander Ng}
\date{Friday, March 28, 2025}

\maketitle

IDK why but he starts off with this example

\section{Quadrature Formula Example}

Find the constants $c_0, c_i$ and $x$ such that the quadrature formula

\begin{equation*}
  \int_{-1}^{0} f(x) \, dx = c_0 f(-1) + c_i f(x_1)
.\end{equation*}

has the highest degree of precision possible.

\uline{ans.}
 
\renewcommand{\arraystretch}{3}
\begin{center}
  \begin{tabular}{c|c}
    Function $f$ & Equation \\
    \hline
    $f(x) = 1$ & $\displaystyle \int_{-1}^{0} 1 \, dx = c_0 + c_i$ \\
    $f(x) = x$ & $\displaystyle \int_{-1}^{0} x \, dx = \left. \frac{x^2}{2} \right|_{-1}^0 = c_0 (-1) + c_i x_1 = -\frac{1}{2}$ \\
    $f(x) = x^2$ & $\displaystyle \int_{-1}^{0} x^2 \, dx = c_0 + c_i x_1^2 = \frac{1}{3}$ \\
  \end{tabular}
\end{center}

Next, we do some substutitions:

\begin{enumerate}
\item substitute (1) into (2) to eliminate $c_0$
\item substitute (1) into (3) to eliminate $c_0$
\end{enumerate}

We now have $2$ equations for $c_i + x_1$.

Solve: $c_0 = \frac{1}{4}, c_i = \frac{3}{4}, x_i = -\frac{1}{3}$

Could it be exact for cubics?

LHS = $\int_{-1}^{0} x^3 \, dx = -\frac{1}{4}$

RHS = $-\underbrace{c_0}_{\frac{1}{4}} +
\underbrace{c_i}_{\frac{3}{4}}\underbrace{x_i}_{-\frac{1}{3}}^3 \neq \text{LHS} $

So, no, it cannot be exact for cubics.

\section{Legendre Polynomials}

This approach can be used to obtain the nodes and coefficients for larger $n$,
but Legendre polynomials can be used to obtain them more easily.

The Legendre Polynomials are defined according to the following two properties:

\begin{enumerate}
\item $P_n(x)$ is a polynomial of degree $n$.\\
  $\implies P_0(x) = 1$ 
\item $\int_{-1}^{1} P(x)P_n(x) \, dx = 0$ whenever $P(x)$ is a polynomial of
  degree less than $n$.
\end{enumerate}

The first few Legendre polynomials are

\renewcommand{\arraystretch}{1.25}
\begin{center}
  \begin{tabular}{c|c}
    $P_n(x)$ & \\ \hline
    $P_0(x)$ & $1$ \\
    $P_1(x)$ & $x$ \\
    $P_2(x)$ & $x^2-\frac{1}{3}$ \\
    $P_3(x)$ & $x^3-\frac{3}{5}x$ \\
    $P_4(x)$ & $x^4-\frac{6}{7}x^2+\frac{3}{35}$ \\
  \end{tabular}
\end{center}

Some properties:

\begin{itemize}
\item The roots of these polynomials are distinct
\item The roots of these polynomials lie in $(-1, 1)$
\item The $P_n$'s are symmetrical about the origin\\
  $\implies$ the roots are symmetrical about the origin
\item The roots of the $n^{th}$ degree Legendre polynomial have the property
  that they are the nodes needed to produce an integral approximation formula
  that gives the exact result for any polynomial of degree less than $2n$.
\end{itemize}

\thm Suppose $x_1, x_2, \dots, x_n$ are the roots of the $n^{th}$ degree
Legendre polynomial $P_n(x)$ and that for each $i=1, 2, \dots, n$ the numbers
$c_i$ are defined by

\begin{equation*}
  c_i = \int_{-1}^{1} \prod_{\substack{j = 1 \\ j \ne i}}^{n} \frac{x-x_j}{x_i-x_j} \, dx
.\end{equation*}

If $P(x)$ is any polynomial of degree less than $2n$ then

\begin{equation*}
  \int_{-1}^{1} P(x) \, dx = \sum_{i=1}^{n} c_i P(x_i)
.\end{equation*}

\end{document}
